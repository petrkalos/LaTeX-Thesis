\chapter{Διόρθωση Παραμόρφωσης Εικόνας από Ευρυγώνιους Φακούς}
\label{chapter:chap2}

Σε αυτό το κεφάλαιο γίνεται μία εισαγωγή στην παραμόρφωση που προκαλείται από την χρήση ευρυγώνιων φακών, αναλύονται τα μοντέλα προοπτικής προβολής και προβολής ευρυγώνιων φακών και παρουσιάζεται ο προτεινόμενος αλγόριθμος διόρθωσης.

\section{Εισαγωγή}
\label{section:sect21}
\indent
Όπως αναφέρθηκε στην ενότητα~\ref{section:sect11}, οι ευρυγώνιοι φακοί επιτρέπουν την λήψη εικόνων με μεγαλύτερο \ac{FoV} σε σχέση με τους συμβατικούς φακούς. Αυτό επιτυγχάνεται με την απεικόνιση σε μία σφαιρική, και όχι σε μία επίπεδη, επιφάνεια. Όμως, η απεικόνιση σε μία σφαιρική επιφάνεια έχει ως αποτέλεσμα την παραμόρφωση της παραγόμενης εικόνας. Αυτού του είδους η παραμόρφωση είναι γνωστή και ως \textsl{barrel distortion} \cite{imagedistortion}, όπου η μεγέθυνση της εικόνας μειώνεται καθώς αυξάνεται η απόσταση από τον οπτικό άξονα. Ως αποτέλεσμα, οι ευθείες γραμμές κυρτώνονται και οι αποστάσεις μεταξύ των αντικειμένων υφίστανται στρέβλωση. Οι ευρυγώνιοι φακοί εκμεταλλεύονται αυτή την παραμόρφωση για την απεικόνιση μίας ευρείας επιφάνειας του αντικειμένου στην πεπερασμένη επιφάνεια της εικόνας.

\begin{figure}
\centering
\includegraphics[width=3in, height=1.8in]{Chapter2/figures/figure1.eps}
\caption{Εικόνα που έχει ληφθεί με χρήση ευρυγώνιου φακού.}
\label{figure:fig21}
\end{figure}

\indent
Το σχήμα~\ref{figure:fig21} παρουσιάζει μία εικόνα η οποία έχει ληφθεί με χρήση ευρυγώνιου φακού. Όπως παρατηρούμε, η παραμόρφωση της εικόνας, υπό την μορφή ενός είδους κύρτωσης, είναι εμφανής και ιδιαιτέρως έντονη στις γωνίες αυτής. Λόγω αυτής της παραμόρφωσης, οι ευρυγώνιοι φακοί δεν μπορούν να χρησιμοποιηθούν σε εφαρμογές εάν δεν ληφθεί ειδική μέριμνα. Επομένως, πριν την προβολή της εικόνας στον τελικό χρήστη, θα πρέπει να μεσολαβεί κάποιο στάδιο διόρθωσης, όπου θα αφαιρείται η παραμόρφωση και η εικόνα θα μετασχηματίζεται στο μοντέλο της κεντρικής προοπτικής προβολής.

\section{Κεντρική Προοπτική Προβολή και Μοντέλο Προβολής Ευρυγώνιων Φακών}
\label{section:sect22}
\indent
Το μαθηματικό μοντέλο της \textsl{κεντρικής προοπτικής προβολής} βασίζεται στην υπόθεση ότι η γωνία πρόσπτωσης μίας ακτίνας από ένα σημείο είναι ίση με την γωνία μεταξύ της ακτίνας και του οπτικού άξονα (ο άξονας απεικονίζεται με πορτοκαλί χρώμα) στο επίπεδο της εικόνας. Το σχήμα~\ref{figure:fig22a}, \cite{Schwalbe05}, παρουσιάζει το μοντέλο προβολής για δύο ακτίνες, με γωνίες πρόσπτωσης \(\alpha_{1}\) και \(\alpha_{2}\). Από το σχήμα, γίνεται εμφανές ότι σημεία του αντικειμένου με γωνία πρόσπτωσης που προσεγγίζει τις \(90^{o}\) προβάλλονται σε άπειρη απόσταση από το πρωτεύον σημείο\footnote{\small Το πρωτεύον σημείο είναι το σημείο από το οποίο διέρχονται όλες οι ακτίνες.} (\textsl{principal point}). Η ακτίνα με το κόκκινο χρώμα αποτελεί ένα παράδειγμα ακτίνας με γωνία πρόσπτωσης ίση με \(90^{o}\), η οποία δεν προβάλλεται στο επίπεδο της εικόνας. Αυτό έχει ως αποτέλεσμα τον περιορισμό των τιμών του \ac{FoV} σε γωνίες οι οποίες δεν αποκλίνουν πολύ από τον οπτικό άξονα.

\begin{figure}
\centering
\includegraphics[width=0.5\textwidth]{Chapter2/figures/figure2a.eps}
\caption{Μοντέλο Κεντρικής Προοπτικής Προβολής.}
\label{figure:fig22a}
\end{figure}

\indent
Όπως και στο μοντέλο των πανοραμικών εικόνων \cite{Schneider}, η γεωμετρία των εικόνων που λαμβάνονται με χρήση ευρυγώνιων φακών δεν ακολουθεί το μοντέλο της κεντρικής προοπτικής προβολής. Επομένως, οι μαθηματικές ιδιότητες αυτού του μοντέλου δεν μπορούν να εφαρμοσθούν στο μοντέλο των ευρυγώνιων φακών. Για να επεκταθεί το εύρος των τιμών που μπορεί να λάβει η γωνία θέασης ενός φακού και για την πλήρη προβολή ημισφαιρίων στο επίπεδο της εικόνας, η ύπαρξη ενός νέου μοντέλου προβολής κρίνεται αναγκαία.

\begin{figure}
\centering
\includegraphics[width=0.5\textwidth]{Chapter2/figures/figure2b.eps}
\caption{Μοντέλο Προβολής Ευρυγώνιων Φακών.}
\label{figure:fig22b}
\end{figure}

\indent
Το μοντέλο προβολής ευρυγώνιων φακών\footnote{\small Στην βιβλιογραφία συναντάται και ως \textsl{fisheye projection model}.} βασίζεται στην παραδοχή ότι, στην ιδανική περίπτωση, η γωνία πρόσπτωσης είναι ανάλογη της απόστασης μεταξύ του σημείου της εικόνας και του πρωτεύοντος σημείου. Το μοντέλο αυτό παρουσιάζεται στο σχήμα~\ref{figure:fig22b}, όπου ισχύει η σχέση \(\frac{d_{1}}{d_{2}} = \frac{\alpha_{1}}{\alpha_{2}}\). Χρησιμοποιώντας αυτό το μοντέλο, οι εισερχόμενες ακτίνες διαθλώνται πιο κοντά στον οπτικό άξονα, επεκτείνοντας έτσι το \ac{FoV} των φακών.\\
\indent
Προκειμένου να είναι δυνατή η ανακατασκευή της προβολής του σημείου ενός αντικειμένου στο επίπεδο της ημισφαιρικής εικόνας, θα πρέπει οι συντεταγμένες του αντικειμένου και της εικόνας να αναφέρονται στο ίδιο σύστημα συντεταγμένων. Για την αντιστοίχιση των συντεταγμένων \((i, j)\) ενός σημείου από τον δισδιάστατο χώρο της κεντρικής προοπτικής προβολής στις συντεταγμένες \((x, y)\) του σημείου στον τρισδιάστατο, ευρυγώνιο, χώρο, θα πρέπει πρώτα να υπολογισθούν οι συντεταγμένες \((X_{c}, Y_{c}, Z_{c})\) της προβολής του σημείου \((i, j)\) στο τρισδιάστατο σύστημα συντεταγμένων της \textsl{camera}. Αυτό επιτυγχάνεται εφαρμόζοντας τον ακόλουθο πίνακα περιστροφής (\textsl{rotation matrix}):

\[
\begin{bmatrix}
X_{c} \\
Y_{c} \\
Z_{c} \\
\end{bmatrix}
=
\begin{bmatrix}
r_{11} & r_{12} & r_{13} \\
r_{21} & r_{22} & r_{23} \\
r_{31} & r_{32} & r_{33} \\
\end{bmatrix}
\times
\begin{bmatrix}
i \\
j \\
1 \\
\end{bmatrix}
\]
\noindent
Το σχήμα~\ref{figure:fig23} απεικονίζει γραφικά τις ιδιότητες της προβολής όταν χρησιμοποιούνται ευρυγώνιοι φακοί. Εάν υποθέσουμε ότι χρησιμοποιείται ένας ευρυγώνιος φακός με γωνία ίση με \(180^{o}\) και ότι η ακτίνα από ένα σημείο του αντικειμένου έχει γωνία πρόσπτωσης ίση με \(90^{o}\), τότε το σημείο προβάλλεται στο εξωτερικό της κυκλικής ευρυγώνιας εικόνας. Αυτό έχει ως αποτέλεσμα το σημείο που προκύπτει να έχει την μέγιστη απόσταση από τον οπτικό άξονα.
\begin{figure}
\centering
\includegraphics[width=4in, height=4in]{Chapter2/figures/figure3.eps}
\caption{Γεωμετρική απεικόνιση της σχέσης μεταξύ ενός σημείου του αντικειμένου και του αντίστοιχου σημείου της εικόνας στο σύστημα συντεταγμένων της \textsl{camera}.}
\label{figure:fig23}
\end{figure}
Η σχέση μεταξύ της γωνίας πρόσπτωσης και της απόστασης του αντίστοιχου σημείου από το πρωτεύον σημείο είναι σταθερή για όλη την εικόνα. Επομένως, ο λόγος που περιγράφεται από την ακόλουθη σχέση μπορεί να χρησιμοποιηθεί ως βασική εξίσωση για την ευρυγώνια προβολή:
\begin{equation}
\label{equation:eqtn21}
\frac{\alpha}{r}=\frac{90^{o}}{R}, r = \sqrt{x^{2} + y^{2}},
\end{equation}
\noindent
όπου \(\alpha\) είναι η γωνία πρόσπτωσης, \(r\) η απόσταση μεταξύ του σημείου της εικόνας και του οπτικού άξονα, \(R\) η ακτίνα της εικόνας και \(x, y\) είναι οι συντεταγμένες στον χώρο της \textsl{wide angle} εικόνας.\\
\indent
Με χρήση κάποιων αλγεβρικών μετασχηματισμών \cite{Schwalbe05}, οι εξισώσεις που περιγράφουν την προβολή στο επίπεδο της εικόνας όταν χρησιμοποιούνται ευρυγώνιοι φακοί είναι οι ακόλουθες:

\begin{subequations}
\label{equation:eqtn22}
\begin{align}
x&=\frac{\frac{2R}{\pi}\cdot \arctan(\frac{\sqrt{X_{c}^{2} + Y_{c}^{2}}}{Z_{c}})}{\sqrt{(\frac{Y_{c}}{X_{c}})^{2} + 1}} + d_{x} + x_{h}\label{equation:eqtn22a}\\
y&=\frac{\frac{2R}{\pi}\cdot \arctan(\frac{\sqrt{X_{c}^{2} + Y_{c}^{2}}}{Z_{c}})}{\sqrt{(\frac{Y_{c}}{X_{c}})^{2} + 1}} + d_{y} + y_{h},\label{equation:eqtn22b}
\end{align}
\end{subequations}
\noindent
όπου (\(X_{c}, Y_{c}, Z_{c}\)) είναι οι συντεταγμένες του σημείου στον τρισδιάστατο χώρο των συντεταγμένων της \textsl{camera}, \(d_{x}, d_{y}\) είναι παράγοντες που μοντελοποιούν τις παραμέτρους της παραμόρφωσης που εισάγει ο εκάστοτε φακός, \(x_{h}, y_{h}\) είναι οι συντεταγμένες του πρωτεύοντος σημείου και \(R\) είναι η ακτίνα της εικόνας. Οι σχέσεις \eqref{equation:eqtn22} παρέχουν μία μέθοδο για την μετατροπή των συντεταγμένων (\(X_{c},Y_{c},Z_{c}\)) ενός αντικειμένου από τον τρισδιάστατο χώρο της κεντρικής προοπτικής προβολής στον διδιάστατο χώρο των ευρυγώνιων φακών. Αυτή η διαδικασία ονομάζεται \textsl{αντίστροφη απεικόνιση - inverse mapping}. Οι προαναφερθείσες σχέσεις μπορούν να διασπαστούν σε στοιχειώδεις συναρτήσεις, όπου πλέον ενσωματώνονται και οι παράγοντες που μοντελοποιούν την παραμόρφωση που εισάγουν οι φακοί. Οι σχέσεις αυτές είναι οι ακόλουθες:

\begin{subequations}
\label{equation:eqtn23}
\begin{align}
d &=\sqrt{X^{2}_{c} + Y^{2}_{c}}\label{equation:eqtn23a}\\
D_{u} &= \frac{d}{Z_{c}}\label{equation:eqtn23b}\\
R_{u} &= \arctan{D_{u}}\label{equation:eqtn23c}\\
P &=k_{1}\cdot R^{4}_{u} + k_{2}\cdot R^{3}_{u} + k_{3}\cdot R^{2}_{u} + k_{4}\cdot R_{u} + k_{5}\label{equation:eqtn23d}\\
x &=\frac{P}{d}\cdot X_{c} + x_{h}\label{equation:eqtn23e}\\
y &= \frac{P}{d}\cdot Y_{c} + y_{h},\label{equation:eqtn23f}
\end{align}
\end{subequations}
\noindent
όπου \(k_{i}\) είναι παράμετροι που εξαρτώνται από τον τύπο του φακού που χρησιμοποιείται. Οι σχέσεις \eqref{equation:eqtn23} μοντελοποιούν την παραμόρφωση που εισάγεται από την χρήση ενός ευρυγώνιου φακού, χρησιμοποιώντας τις παραμέτρους του αντίστοιχου αισθητήρα \(k[]\). Δοθέντων των συντεταγμένων (\(X_{c}, Y_{c}, Z_{c}\)) ενός σημείου, χρησιμοποιούμε τις σχέσεις \eqref{equation:eqtn23} ώστε να υπολογίσουμε τις αντίστοιχες συντεταγμένες του σημείου στον χώρο των ευρυγώνιων φακών.

\section{Προτεινόμενος Αλγόριθμος Διόρθωσης}
\label{section:sect23}
\indent
Ο προτεινόμενος αλγόριθμος λαμβάνει ως είσοδο μία εικόνα, που έχει ληφθεί από έναν ευρυγώνιο φακό και έχει υποστεί παραμόρφωση, όπως επίσης και την τιμή της παραμέτρου \ac{FoV} και παράγει ως έξοδο την εικόνα όπου πλέον έχει αφαιρεθεί η παραμόρφωση. Ο χρήστης καθορίζει το κέντρο του \textsl{πεδίου ενδιαφέροντος}, \ac{ROI}, με μέγεθος ίσο με \(1280x960\) \textsl{pixels} και η εικόνα που παράγεται έχει μέγεθος εικόνας τύπου \textsl{VGA}, δηλαδή \(640x480\) \textsl{pixels}. Το σχήμα~\ref{figure:fig24} παρουσιάζει μία ενδεικτική χρήση του αλγορίθμου. Όπως παρατηρούμε, μικρότερες τιμές της παραμέτρου \ac{FoV} έχουν ως αποτέλεσμα την ύπαρξη μεγέθυνσης, \textsl{zoom}, στην εικόνα που παράγεται.

\begin{figure}
\centering
\includegraphics[width=0.55\textwidth]{Chapter2/figures/figure4.eps}
\caption{Ο αλγόριθμος διόρθωσης για τιμές της παραμέτρου \textsl{FoV} ίσες με \textsl{FoV} = $60^{o}$ και \textsl{FoV} = $8^{o}$.}
\label{figure:fig24}
\end{figure}
Για να είναι δυνατή η διόρθωση της παραμόρφωσης, αρχικά θα πρέπει να γίνει η αντίστροφη απεικόνιση για τα \textsl{pixels} εντός του \textsl{πεδίου ενδιαφέροντος}. Για το σκοπό αυτό χρησιμοποιούνται οι σχέσεις \eqref{equation:eqtn23}. Όπως παρατηρούμε, οι εν λόγω σχέσεις δεν παράγουν ακέραιες τιμές για τις συντεταγμένες των \textsl{pixels} στο πεδίο της εικόνας. Επομένως, καθώς δεν υπάρχει απευθείας αντιστοίχιση των \textsl{pixels} του ενός χώρου στα \textsl{pixels} του άλλου χώρου, θα πρέπει να χρησιμοποιηθεί κάποιο σχήμα παρεμβολής για τον υπολογισμό των τιμών του \textsl{pixels}, χρησιμοποιώντας τιμές από γειτονικά \textsl{pixels}, τα οποία βρίσκονται σε ακέραιες θέσεις στον χώρο. Για τον υπολογισμό των τιμών του εκάστοτε \textsl{pixel} που βρίσκεται σε μη ακέραια θέση χρησιμοποιούμε το σχήμα παρεμβολής \textsl{Βicubic Ιnterpolation} \cite{Keyes81}.\newline \indent
Αυτό το σχήμα παρεμβολής αποτελεί μία μέθοδο με μεγάλη ανοχή σε σφάλματα αλλά ταυτόχρονα και με μεγάλες υπολογιστικές απαιτήσεις. Η μέθοδος αυτή χρησιμοποιεί πολυωνυμικές συναρτήσεις τρίτου βαθμού ώστε να προσεγγιστούν οι τιμές των ενδιάμεσων σημείων. Καθώς το πολυώνυμο που χρησιμοποιείται είναι τρίτου βαθμού, απαιτούνται τέσσερεις κόμβοι για την παρεμβολή, έστω \(C_{i}, i \in 1\dots4\). Δοθέντων των κόμβων της παρεμβολής, \(C_{i}\), όπου η συνάρτηση παρεμβολής \(g()\) είναι ίση με την, άγνωστη, παρεμβαλλόμενη συνάρτηση \(f()\), η διαδικασία παρεμβολής για την τιμή της συνάρτησης \(f()\) σε ένα σημείο \(x\) δύναται να περιγραφεί από τις ακόλουθες σχέσεις:

\begin{subequations}
\label{equation:eqtn24}
\begin{align}
\ g(x)&=C_{1}\cdot U_{1}(s) + C_{2}\cdot U_{2}(s) + C_{3}\cdot U_{3}(s) + C_{4}\cdot U_{4}(s)\label{equation:eqtn24a}\\
U_{1}&=\frac{-s^{3} + 2s^{2} - s}{2}\label{equation:eqtn24b}\\
U_{2}&=\frac{3s^{3} - 5s^{2} + 2}{2}\label{equation:eqtn24c}\\
U_{3}&=\frac{-3s^{3} + 4s^{2} + s}{2}\label{equation:eqtn24d}\\
U_{4}&=\frac{s^{3} - s^{2}}{2}\label{equation:eqtn24e}
\end{align}
\end{subequations}
\noindent
όπου το σημείο \(x\) είναι τέτοιο ώστε \(x_{1} \leq x_{2} \leq x \leq x_{3} \leq x_{4} \) και η παράμετρος \(s\) ισούται με \( s = x - x_{2} \).

\begin{figure}
\centering
\includegraphics[width=0.7\textwidth]{Chapter2/figures/figure5.eps}
\caption{H αντίστροφη απεικόνιση και το σχήμα παρεμβολής στην 4x4 γειτονιά.}
\label{figure:fig25}
\end{figure}
Το σχήμα~\ref{figure:fig25} παρουσιάζει την \textsl{αντίστροφη απεικόνιση} και την μέθοδο παρεμβολής που χρησιμοποιείται για παρεμβολή στις δύο διαστάσεις. Η παρεμβολή στις δύο διαστάσεις επιτυγχάνεται παρεμβάλλοντας πρώτα τα σημεία στην οριζόντια διάσταση και έπειτα στην κάθετη διάσταση. Για την παρεμβολή χρησιμοποιούνται τα \textsl{pixels} που βρίσκονται εντός ενός παραθύρου διαστάσεων \(4x4\) γύρω από το επιθυμητό σημείο. Η συνάρτηση παρεμβολής \(G()\) στο σημείο με συντεταγμένες \((x, y)\) παρουσιάζεται στις ακόλουθες σχέσεις:

\begin{subequations}
\label{equation:eqtn25}
\begin{align}
G(x,y)&=g_{1}(x)\cdot V_{1}(t) + g_{2}(x)\cdot V_{2}(t) + g_{3}(x)\cdot V_{3}(t) + g_{4}(x)\cdot V_{4}(t)\label{equation:eqtn25a}\\
V_{1}(t)&=\frac{-s^{3} + 2s^{2} - s}{2}\label{equation:eqtn25b}\\
V_{2}(t)&=\frac{3s^{3} - 5s^{2} + 2}{2}\label{equation:eqtn25c}\\
V_{3}(t)&=\frac{-3s^{3} + 4s^{2} + s}{2}\label{equation:eqtn25d}\\
V_{4}(t)&=\frac{s^{3} - s^{2}}{2}\label{equation:eqtn25e}
\end{align}
\end{subequations}
\noindent
όπου \(t = x - \lfloor x \rfloor\) και \(g_{k}(x)\) είναι η τιμή που προκύπτει από την παρεμβολή στην οριζόντια διάσταση για τις γραμμές \(\lfloor y \rfloor - 1\), \(\lfloor y \rfloor\), \(\lfloor y \rfloor + 1\) και \(\lfloor y \rfloor + 2\).\\
\indent
Παρόλο που υπήρχε η δυνατότητα χρήσης και άλλων, απλούστερων, τεχνικών παρεμβολής, όπως παρεμβολή βάσει του κοντινότερου γειτονικού\footnote{\small Nearest Neighbor} \textsl{pixel} ή διγραμμική παρεμβολή\footnote{\small Bilinear Interpolation}, οι υψηλές απαιτήσεις της εφαρμογής όσον αφορά στον λόγο \textsl{Σήματος προς Θόρυβο} (\textsl{Signal to Noise ratio}), ειδικότερα για μικρές τιμές της παραμέτρου \ac{FoV}, καθιστούν τη μέθοδο αυτή ως την βέλτιστη για την επίτευξη του επιθυμητού αποτελέσματος. Επίσης, υπολογιστικά πολυπλοκότερες μέθοδοι, όπως η μέθοδος \textsl{Elliptical Weighted Average (EWA)} \cite{GreeneHeckbert86}, απορρίφθηκαν καθώς το κόστος αυτών ήταν μεγαλύτερο από αυτό που απαιτούνταν από την εφαρμογή ενώ ενδεχομένως καθιστούσαν και την διόρθωση της εικόνας σε πραγματικό χρόνο μη εφικτή.\\
\indent
Όπως αναφέρθηκε και παραπάνω, τα κύρια στάδια του αλγορίθμου, δηλαδή η \textsl{αντίστροφη απεικόνιση} και η εφαρμογή του σχήματος παρεμβολής, εκτελούνται σε μία εικόνα διαστάσεων \(1280x960\). Η εκτέλεση αυτών των σταδίων υπολογισμού σε μία εικόνα μεγαλύτερων διαστάσεων από την εικόνα εξόδου συντελεί στην διατήρηση της πληροφορίας που εμπεριέχεται σε περιοχές με υψηλή χωρητική συχνότητα. Το τελευταίο στάδιο στην επεξεργασία της εικόνας είναι η δειγματοληψία προς τα κάτω της εικόνας ώστε από την αρχική ανάλυση \(1280x960\) να μεταβούμε σε ανάλυση \textsl{VGA} για την εικόνα εξόδου. Αυτό επιτυγχάνεται με την εφαρμογή ενός \textsl{5-tap} βαθυπερατού φίλτρου πρώτα στην κάθετη και έπειτα στην οριζόντια κατεύθυνση με συντελεστές \(0, 0.171887, 0.5\), \(0.171887, 0\). Η εφαρμογή του φίλτρου σε κάθε μία κατεύθυνση έχει ως αποτέλεσμα την δειγματοληψία προς τα κάτω κατά έναν παράγοντα ίσο με \(2\) ώστε, συνολικά, το μέγεθος της εικόνας να υποτετραπλασιάζεται. Αντί της εφαρμογής του φίλτρου, η κλιμάκωση προς τα κάτω θα ήταν δυνατή με επιλογή κάθε δεύτερου \textsl{pixel} από την εικόνα που παράγεται μετά το στάδιο της παρεμβολής. Αυτό το σχήμα δεν θα παρήγαγε το επιθυμητό αποτέλεσμα ενώ θα στερούσε από την εφαρμογή και μία θετική παρενέργεια του βαθυπερατού φίλτρου, η οποία είναι η απομάκρυνση των υψηλής συχνότητας, τεχνητών \textsl{pixels}\footnote{\small Στην βιβλιογραφία αναφέρονται και ως artifacts}. Σε αυτό το σημείο, θα πρέπει να αναφερθεί ότι η απομάκρυνση αυτών των \textsl{pixels} δεν αποτελεί πρωτεύων μέλημα στην σχεδίαση του αλγορίθμου αλλά το φίλτρο εφαρμόζεται ως μία καλύτερη εναλλακτική λύση για την επιθυμητή δειγματοληψία της εικόνας.\\
\indent
Συνοψίζοντας, τα στάδια επεξεργασίας του προτεινόμενου αλγορίθμου είναι τα ακόλουθα:

\begin{itemize}

\item{Αντίστροφη Απεικόνιση: Σε αυτό το στάδιο τα \textsl{pixels} της εικόνας εξόδου αντιστοιχίζονται σε \textsl{pixels} στο πεδίο των ευρυγώνιων φακών.}

\item{Παρεμβολή: Το στάδιο της παρεμβολής χρησιμοποιείται ώστε να προσεγγιστούν οι τιμές των \textsl{pixels} της εικόνας εξόδου τα οποία αντιστοιχίζονται σε κλασματικές θέσεις κατά την αντίστροφη απεικόνιση.}

\item{Βαθυπερατό Φίλτρο: Το τελευταίο στάδιο επεξεργασίας του αλγορίθμου είναι η εφαρμογή ενός βαθυπερατού φίλτρου ώστε να γίνει η προς τα κάτω δειγματοληψία της εικόνας.}

\end{itemize}
\indent
Το σχήμα~\ref{figure:fig26} παρουσιάζει έναν ενδεικτικό, υψηλού επιπέδου, ψευδοκώδικα για τον αλγόριθμο διόρθωσης της παραμόρφωσης που προκαλείται από την χρήση ευρυγώνιων φακών.
\begin{figure}
\centering
\begin{small}
\begin{algorithmic}[1]
\STATE \COMMENT{\textbf{Input}: The frames (in the wide-angle space) to be corrected}
\STATE \COMMENT{\textbf{Output}: The corrected frames (in the perspective space)}
\FORALL{frames}
\FORALL{pixels in the output frame}
\STATE Compute the corresponding fractional position in the input frame ({\textbf{ InverseMapping()}})
\STATE Interpolate the pixel value at that fractional position ({\textbf{ BicubicInterpolation()}})
\ENDFOR
\STATE Apply a $2$-D low-pass filter to resize the output frame ({\textbf{ LPF()}})
\ENDFOR
\end{algorithmic}
\end{small}
\caption{\label{figure:fig26} Ψευδοκώδικας για τον αλγόριθμο διόρθωσης (αρχική έκδοση).}
\end{figure}
