\thispagestyle{plain}
\begin{abstract}
Οι ευρυγώνιοι φακοί, \textsl{wide-angle} ή \textsl{fisheye lenses}, χρησιμοποιούνται σε επιστημονικές εφαρμογές ή εφαρμογές εικονικής πραγματικότητας (\textsl{virtual reality}) ώστε να επεκτείνουν το πεδίο θέασης, \ac{FoV}, των συμβατικών φωτογραφικών μηχανών. Λόγω του μεγαλύτερου πεδίου θέασης, οι εικόνες που λαμβάνονται από τέτοιου είδους φακούς εμφανίζουν κάποιου είδους παραμόρφωση. Η διόρθωση παραμόρφωσης που προκαλείται από την χρήση ευρυγώνιων φακών είναι μία εφαρμογή επεξεργασίας εικόνας (\textsl{image warping application}), η οποία χρησιμοποιείται ώστε να μετατρέψει την παραμορφωμένη εικόνα από το μοντέλο των ευρυγώνιων φακών στο μοντέλο της κεντρικής προοπτικής προβολής (\textsl{central perspective space}). Η εφαρμογή χαρακτηρίζεται από ένα μη γραμμικό, συνεχούς ροής, πρότυπο προσπέλασης της μνήμης το οποίο καθιστά το εύρος ζώνης της κύριας μνήμης έναν παράγοντα ο οποίος επηρεάζει σημαντικά την απόδοση της εφαρμογής.\newline
\indent
Ο επεξεργαστής \textsl{Cell}, \ac{CBE}, είναι ένας μη συμβατικός και ετερογενής πολύ-επεξεργαστής. Η αρχιτεκτονική του επεξεργαστή \textsl{Cell} επιτρέπει την επιτάχυνση εφαρμογών με μεγάλο βαθμό παραλληλισμού τόσο σε επίπεδο νημάτων (\textsl{thread-level parallelism}) όσο και σε επίπεδο δεδομένων (\textsl{data-level parallelism}).\newline
\indent
Η παρούσα διπλωματική εργασία παρουσιάζει την υλοποίηση, βελτιστοποίηση και αξιολόγηση ενός αλγορίθμου για την διόρθωση παραμόρφωσης εικόνας, που προκαλείται από την χρήση ευρυγώνιων φακών, στον επεξεργαστή \textsl{Cell} και σε πραγματικό χρόνο. Για την επίλυση των προβλημάτων του συστήματος μνήμης εφαρμόζουμε βελτιστοποιήσεις σε επίπεδο πηγαίου κώδικα όπως η τεχνική του \textsl{tiling} για την καλύτερη εκμετάλλευση της \textsl{on-chip} μνήμης των \acp{SPE} και για την μεγιστοποίηση της επαναχρησιμοποίησης των δεδομένων εντός του \textsl{frame}. Οι προτεινόμενες βελτιστοποιήσεις καθιστούν την διόρθωση παραμόρφωσης εικόνας στον επεξεργαστή \ac{CBE} μία διαδικασία που μπορεί να πραγματοποιηθεί σε πραγματικό χρόνο ενώ επιτυγχάνεται επιτάχυνση (\textsl{speedup}) ίση με \(7.27x\), σε σύγκριση με έναν επεξεργαστή \textsl{Intel Core2 Duo}.
\end{abstract}
\vspace{0.5in}
\begin{Large}
\textbf{Λέξεις Κλειδιά:}\\
\end{Large}
Παραμόρφωση Εικόνας, Πραγματικός Χρόνος, Παράλληλος Προγραμματισμός,\\ Υπολογισμός Stencil, Cell B.E., Ετερογενείς Επεξεργαστές Πολλαπλών Πυρήνων 