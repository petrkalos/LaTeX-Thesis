\thispagestyle{plain}
\begin{abstract}

\indent Η συμπίεση του ψηφιακού βίντεο είναι απαραίτητη προϋπόθεση για να μπορούμε να αποθηκεύσουμε, αναπαράγουμε και να αποστείλουμε βίντεο. Λόγο του πλεονάζοντος αριθμού δεδομένων ακόμα και με τους σημερινούς ηλεκτρονικούς υπολογιστές δε θα μας επιτρέπονταν να κάνουμε τίποτα από αυτά αν δεν υπήρχε η συμπίεση. Αυτή η διπλωματική εστιάζει στην βελτίωση των τεχνικών συμπίεσης βίντεο με την χρήση στοιχείων της θεωρίας πληροφοριών.

\indent Η διπλωματική αυτή μελετάει την απόδοση της κβαντοποίησης διανυσμάτων στην συμπίεση βίντεο, μια τεχνική που πολύ λίγο έχει δοκιμαστεί στο παρελθόν.
 Ο βασικός αλγόριθμος ήταν ο k-means και έγιναν βελτιστοποιήσεις οπού τελικά πέτυχαν επιδόσεις x$11$ σε σχέση με τον αρχικό κώδικα, πράγμα που μας επέτρεψε να τρέχουν μεγάλα πειράματα σε εύλογο χρονικό διάστημα. Χρησιμοποιήθηκαν εργαλεία της θεωρίας πληροφοριών όπως η έννοια της εντροπίας και της υπό συνθήκη εντροπίας για να μπορέσει να προσεγγιστεί η απόδοση της κβαντοποίησης διανυσμάτων και τέλος ο JM H.264 μετατράπηκε ώστε να υποστηρίζει κβαντοποίηση διανυσμάτων.

\indent Οι μετρήσεις που έγιναν έδειξαν πως ο VQ H.264 πετυχαίνει $20\%$ καλύτερο bitrate κατά μέσο όρο σε σχέση με τον JM H.264 και φαίνεται να υπάρχει χώρος για βελτιστοποιήσεις σε μελλοντικές μελέτες.

\end{abstract}
\vspace{0.5in}
\begin{Large}
\textbf{Λέξεις Κλειδιά:}\\
\end{Large}
Κβαντοποίηση Διανυσμάτων, Εντροπία, Θεωρία Πληροφοριών, Κωδικοποιητές Βίντεο, Κωδικοποιητές Εντροπίας, H.264, k-means 