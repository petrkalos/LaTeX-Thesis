\thispagestyle{plain}
\def\abstractname{Abstract}

\begin{abstract}
\indent Compression of digital video is a prerequisite for being able to store, play and transmit video. Due to the redundancy of video data even today's computers would not allow us to do any of that if there was no compression. This thesis focuses on improving video compression techniques using elements of information theory.

\indent This thesis studies the performance of vector quantization in video compression, a technique that very little has been used before. The main algorithm was the k-means which optimized and performs x$11$ in comparison with the first version of code, which allowed us to run large experiments in a reasonable time. We derive tools and theorems from information theory such us entropy and conditional entropy in order to approach the performance of vector quantization and finally JM H.264 converted in a way to support vector quantization.

\indent Experiments showed that VQ H.264 achieves $20\%$ better bitrate on average compared with JM H.264 and there seems to be space for further optimizations in future studies.


\end{abstract}
\vspace{0.5in}
\begin{Large}
\textbf{Keywords:}\\
\end{Large}
Vector Quantization, Entropy, Information Theory, Video Codec, Entropy Encoding, H.264, k-means 