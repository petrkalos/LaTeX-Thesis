\chapter{Γενική Επισκόπηση του Επεξεργαστή Cell (Cell Broadband Engine)}
\label{chapter:chap3}

Στο παρόν κεφάλαιο γίνεται μία εισαγωγή στον επεξεργαστή \textsl{Cell}. Αρχικά, παρουσιάζονται τα κίνητρα για την δημιουργία του επεξεργαστή, έπειτα παρουσιάζεται η αρχιτεκτονική αυτού και τέλος αναλύονται τα εργαλεία για την ανάπτυξη εφαρμογών στον επεξεργαστή.

\section{Εισαγωγή}
\label{section:sect31}
\indent
Οι αυξανόμενες απαιτήσεις νέων εφαρμογών, όπως εφαρμογές πολυμέσων, σε υπολογιστική δύναμη έχουν οδηγήσει σε μία συνεχή αύξηση των απαιτήσεων για επεξεργαστική ισχύ.\newline \indent
Κατά την προηγούμενη δεκαετία, η έλευση της τεχνολογίας \ac{VLSI} και η συνεχής αλλαγή κλίμακας είχαν ως αποτέλεσμα την αύξηση της απόδοσης των μικροεπεξεργαστών κατά \(50\% - 60\%\). Οι κύριοι παράγοντες που συντελούσαν σε αυτή την αύξηση ήταν η αύξηση της ταχύτητας του ρολογιού, τόσο με την μείωση των επιπέδων λογικής για κάθε κύκλο ρολογιού όσο και με την αλλαγή κλίμακας της τεχνολογίας, όπως επίσης και η εκμετάλλευση του μεγάλου αριθμού των transistors σε κάθε chip, ο οποίος διπλασιαζόταν βάσει του \textsl{Νόμου του Moore}.\newline \indent
Πλέον, η επίτευξη μεγάλης απόδοσης στους μικροεπεξεργαστές με χρήση συμβατικών τεχνικών θα είναι πολύ δύσκολη. Ενώ η ταχύτητα των transistors αυξάνεται δραματικά, η ταχύτητα των καλωδίων μειώνεται. Σε συνδυασμό με την αύξηση του μεγέθους του chip, η οποία παρατηρείται στους νέους μικροεπεξεργαστές, ο χρόνος για να αποσταλεί ένα σήμα θα αυξηθεί σημαντικά αφού η απόσταση την οποία θα μπορεί το σήμα να διανύσει σε έναν κύκλο ρολογιού θα μειωθεί. Έτσι, δεν θα είναι δυνατή τόσο η αύξηση των σταδίων του pipeline, η οποία τα προηγούμενα χρόνια είχε επιφέρει σημαντική βελτίωση στην απόδοση των μικροεπεξεργαστών, όσο και η προσθήκη επιπλέον μονάδων εκτέλεσης καθώς η καθυστέρηση για την μετάδοση ενός σήματος μεταξύ των σταδίων του pipeline και μεταξύ των επιμέρους υπομονάδων θα αυξηθεί. Ο συνδυασμός των ανωτέρω θα έχει άμεση επίδραση στην διεκπεραιωτική ικανότητα (\textsl{throughput}) του επεξεργαστή.\newline \indent
Συν τοις άλλοις, η προαναφερθείσα αύξηση της καθυστέρησης θα αυξήσει και τον χρόνο που απαιτείται για πρόσβαση στις δομικές μονάδες του επεξεργαστή, όπως στους φακέλους καταχωρητών, στην μνήμη cache ακόμη και στο παράθυρο εντολών (\textsl{instruction window}). Καθώς η καθυστέρηση πρόσβασης εξαρτάται από την χωρητικότητα της κάθε μονάδας, για να είναι δυνατή η επίτευξη χαμηλής καθυστέρησης θα πρέπει η χωρητικότητα της κάθε μονάδας να μειωθεί. Αυτός θα είναι ένας παράγοντας που θα επηρεάσει την απόδοση των μικροεπεξεργαστών καθώς δεν θα είναι πλέον δυνατή η επίτευξη μεγαλύτερης απόδοσης μέσω της αύξησης του μεγέθους των δομικών μονάδων. Ως ενδεικτικό παράδειγμα μπορεί να αναφερθεί η εξάρτηση του παραλληλισμού που μπορεί να επιτευχθεί σε επίπεδο εντολής από το μέγεθος του παραθύρου εντολών.\newline \indent
Επίσης, η αύξηση της καθυστέρησης θα έχει άμεση επίδραση στον παραλληλισμό σε επίπεδο εντολών (ILP). Σε συνδυασμό με την αύξηση της συχνότητας του ρολογιού, η κατάσταση που θα μπορεί ο επεξεργαστής να εκμεταλλευτεί για δεδομένο βάθος του pipeline θα μειώνεται. Τέλος, η αύξηση της συχνότητας λειτουργίας του επεξεργαστή έχει ως αποτέλεσμα την αύξηση της καταναλισκόμενης ισχύος και ιδίως της στατικής συνιστώσας αυτής. Το κυρίως πρόβλημα που δημιουργεί η αύξηση της ισχύος είναι η αύξηση της πυκνότητας ισχύος, η οποία γίνεται εντονότερη με την αλλαγή κλίμακας των επεξεργαστών.\newline \indent
Λαμβάνοντας υπόψη τα ανωτέρω, οι σχεδιαστές μικροεπεξεργαστών στράφηκαν στην κατασκευή επεξεργαστών με μεγαλύτερο αριθμό από απλούστερους πυρήνες στο ίδιο \textsl{chip}. Αυτού του είδους οι επεξεργαστές αποκαλούνται επεξεργαστές πολλαπλών πυρήνων - \textsl{multicores}. Οι κυρίαρχες προσεγγίσεις στην  κατασκευή επεξεργαστών πολλαπλών πυρήνων είναι δύο. Η πρώτη είναι η \textsl{Συμμετρική Πολυεπεξεργασία} (\ac{SMP}), όπου σε ένα \textsl{chip} τοποθετούνται πολλαπλοί όμοιοι επεξεργαστές. Αυτή η προσέγγιση υιοθετείται από τις τελευταίες γενεές επεξεργαστών της \textsl{Intel} και της \textsl{AMD}. Η δεύτερη προσέγγιση συνίσταται στην δημιουργία \textsl{tiled} ή \textsl{heterogeneous} αρχιτεκτονικών. Χαρακτηριστικά παραδείγματα αυτής της προσέγγισης αποτελούν οι επεξεργαστές \textsl{Cell} \cite{Hofstee} και \textsl{RAW} \cite{Waingold, Taylor}.
\newline \indent
Ο επεξεργαστής \textsl{Cell} (\ac{CBE}) είναι ένας μικροεπεξεργαστής που δημιουργήθηκε από την συνεργασία των εταιρειών \textsl{Sony, Toshiba} και \textsl{IBM} το 2001 ώστε να αποτελέσει τον επεξεργαστή για την νέα \textsl{game console} της \textsl{Sony}, το \textsl{PlayStation 3}. Ενώ αρχικά σχεδιάστηκε για χρήση σε \textsl{game consoles} και σε συσκευές πολυμέσων για το καταναλωτικό κοινό, είναι αρκετά ευέλικτος ώστε να χρησιμοποιηθεί ως επεξεργαστής γενικού σκοπού. Ταυτόχρονα, υπάρχει μεγάλο ερευνητικό ενδιαφέρον για την χρήση του επεξεργαστή σε μία πληθώρα άλλων εφαρμογών όπως εφαρμογές επιστημονικού υπολογισμού ή εφαρμογές υπερ-υπολογισμού (\textsl{supercomputing}).\newline \indent
Η πρώτη υλοποίηση αυτού αποτελεί έναν \textsl{single-chip} επεξεργαστή πολλαπλών πυρήνων με εννέα επεξεργαστές, οι οποίοι λειτουργούν σε ένα μοντέλο κοινής μνήμης (\textsl{shared memory model}), όπως παρουσιάζεται στο σχήμα~\ref{figure:fig31}. Το χαρακτηριστικό που διαφοροποιεί τον \ac{CBE} από ήδη υπάρχοντες επεξεργαστές είναι ότι ενώ όλοι οι επεξεργαστές διαμοιράζονται την υπάρχουσα μνήμη, αυτοί ειδικεύονται σε δύο κατηγορίες: το \acf{PPE} και το \acf{SPE}. Ο επεξεργαστής \textsl{Cell} αποτελείται από έναν επεξεργαστή τύπου \ac{PPE} και οχτώ επεξεργαστές τύπου \ac{SPE}.

\begin{figure}
\centering
\includegraphics[width=5in, height=3in]{Chapter3/figures/figure1.eps}
\caption{Η αρχιτεκτονική του επεξεργαστή \textsl{Cell}.}
\label{figure:fig31}
\end{figure}
Ο πρώτος τύπος επεξεργαστικού στοιχείου, το \ac{PPE}, εμπεριέχει έναν πυρήνα \textsl{64-bit} PowerPC\circledR \ Architecture\(^{TM}\). Ο πυρήνας ακολουθεί την αρχιτεκτονική \textsl{64-bit} PowerPC και μπορεί να εκτελέσει λειτουργικά συστήματα και εφαρμογές \textsl{32-bit} και \textsl{64-bit}. Ο δεύτερος τύπος επεξεργαστικού στοιχείου, το \ac{SPE}, αποτελεί έναν πυρήνα βελτιστοποιημένο για την εκτέλεση \acf{SIMD} εφαρμογών με μεγάλες απαιτήσεις όσον αφορά στην επεξεργαστική ισχύ.\newline \indent
Τα \acp{SPE} είναι ανεξάρτητα επεξεργαστικά στοιχεία, με τον κάθε πυρήνα να εκτελεί τις δικές του εφαρμογές ή νήματα, και έχουν πλήρη πρόσβαση στην κοινή μνήμη. Μεταξύ των \acp{SPE} και του \ac{PPE} υπάρχει μία αμοιβαία εξάρτηση. Τα \acp{SPE} εξαρτώνται από το \ac{PPE} για την εκτέλεση του λειτουργικού συστήματος, την εκτέλεση διαφόρων κλήσεων συστήματος και στην πλειονότητα των περιπτώσεων για την εκτέλεση του νήματος που ελέγχει τα νήματα που εκτελούνται στα \acp{SPE}. Το \ac{PPE} εξαρτάται από τα \acp{SPE} τα οποία είναι υπεύθυνα για το μεγαλύτερο μέρος της απόδοσης του συστήματος.\newline \indent
Για τον προγραμματιστή κάποιας εφαρμογής, ο \textsl{Cell} αποτελεί έναν \textsl{dual-threaded} επεξεργαστή με οχτώ επιπλέον πυρήνες, οι οποίοι έχουν πρόσβαση στην δική τους τοπική μνήμη. Η ποικιλομορφία του \textsl{Cell}, με τον διαχωρισμό των πυρήνων σε πυρήνες βελτιστοποιημένους για λειτουργίες ελέγχου και πυρήνες βελτιστοποιημένους για υπολογισμούς, είναι το στοιχείο που επιτρέπει την δραματική βελτίωση στην μέγιστη υπολογιστική δυνατότητα που διαθέτει και τον καθιστά περισσότερο αποδοτικό, όσον αφορά στην κατανάλωση ισχύος και στην απαιτούμενη επιφάνεια για την υλοποίηση, έναντι συμβατικών μικροεπεξεργαστών.\newline \indent
Η κυριότερη διαφορά μεταξύ των δύο τύπων πυρήνων στον επεξεργαστή \textsl{Cell} είναι ο τρόπος με τον οποίο πραγματοποιείται η προσπέλαση μνήμης. Το μεν \ac{PPE} προσπελαύνει την κύρια μνήμη (\textsl{effective address space}) με εντολές τύπου \textsl{load, store}, με τα αποτελέσματα να αποθηκεύονται σε έναν ιδιωτικό φάκελο καταχωρητών ενώ δίνεται και η δυνατότητα για \textsl{caching} των δεδομένων. Αυτή η μέθοδος πρόσβασης ομοιάζει με τον τρόπο πρόσβασης στην μνήμη όπως αυτή πραγματοποιείται στους συμβατικούς μικροεπεξεργαστές.\newline \indent
Κάθε \ac{SPE}, σε αντίθεση, προσπελαύνει την κύρια μνήμη μέσω εντολών \acf{DMA}, οι οποίες μεταφέρουν δεδομένα μεταξύ της κύριας μνήμης και μίας τοπικής, ιδιωτικής, μνήμης, η οποία αποκαλείται \acf{LS}. Η τοπική μνήμη προσπελαύνεται απευθείας από τις εντολές προσκόμισης, φόρτωσης και αποθήκευσης ενώ δεν υπάρχει κάποια κρυφή μνήμη \textsl{cache}. Αυτή η ιεραρχία μνήμης τριών επιπέδων (φάκελος καταχωρητών, τοπική μνήμη, κύρια μνήμη), με τις ασύγχρονες εντολές \ac{DMA}, αποτελεί μία ριζική αλλαγή σε σχέση με τις συμβατικές αρχιτεκτονικές και τα προγραμματιστικά μοντέλα καθώς ο παραλληλισμός μεταξύ εκτέλεσης υπολογισμών και μεταφοράς δεδομένων είναι σαφής.\newline \indent
Το κυρίως κίνητρο για την υιοθέτηση αυτού του μοντέλου είναι η μεγάλη καθυστέρηση που υπάρχει για την προσπέλαση της κύριας μνήμης, η οποία έχει αυξηθεί δραματικά στις τελευταίες γενεές επεξεργαστών. Αυτό έχει σαν συνέπεια η απόδοση της εφαρμογής να εξαρτάται από την καθυστέρηση της μνήμης και όχι από την υπολογιστική ικανότητα. Επί παραδείγματι, εάν η εφαρμογή έχει μία αστοχία κατά την πρόσβαση στην μνήμη \textsl{cache}, θα πρέπει να παγώσει την εκτέλεσή της για αρκετούς κύκλους ρολογιού με αποτέλεσμα την μείωση της απόδοσης. Εν αντιθέσει, ο μικρός αριθμός κύκλων ρολογιού που απαιτείται για την έναρξη της εκτέλεσης μίας εντολής \ac{DMA} και η δυνατότητα επικάλυψης υπολογισμών και μεταφοράς δεδομένων με χρήση της τεχνικής του \textsl{multi-buffering} μπορούν να αυξήσουν την απόδοση της εκάστοτε εφαρμογής.\newline \indent
Με αυτόν τον καινοτόμο σχεδιασμό η μικρο-αρχιτεκτονική του επεξεργαστή \ac{CBEA} μπορεί να υποστηρίξει μεγάλη υπολογιστική ικανότητα μέσω της εκμετάλλευσης των παρακάτω:

\begin{itemize}

\item{\textsl{Παραλληλισμός μεταξύ των δεδομένων} (\ac{DLP}) με τις μονάδες \ac{SIMD} που βρίσκονται στα \acp{SPE}.}

\item{\textsl{Παραλληλισμός μεταξύ των εντολών} (\ac{ILP}) με την \textsl{dual-issue} αρχιτεκτονική των \textsl{pipelines}.}

\item{\textsl{Παραλληλισμός μεταξύ των νημάτων} (\ac{TLP}) μέσω των πολλαπλών \acp{SPU}.}

\end{itemize}
\indent
Επίσης, η σχεδίαση του συστήματος της μνήμης παρέχει έναν μεγάλο βαθμό από ταυτόχρονες μεταφορές δεδομένων, οι οποίες πραγματοποιούνται υπό τον έλεγχο της εφαρμογής για την υποστήριξη μεγαλύτερης απόδοσης του συστήματος μνήμης. Με αυτά τα χαρακτηριστικά, η εκάστοτε εφαρμογή μπορεί να έχει τον πλήρη έλεγχο των διαθέσιμων πόρων της αρχιτεκτονικής ώστε να επιτευχθεί η μέγιστη δυνατή απόδοση.

\subsection[3.1.1 Αντιμετώπιση των κύριων περιοριστικών παραγόντων της απόδοσης]{Αντιμετώπιση των κύριων περιοριστικών παραγόντων της απόδοσης}
\label{subsection:sub311}
\indent
Η απόδοση των σύγχρονων επεξεργαστών καθορίζεται από τρεις σημαντικούς παράγοντες: την συχνότητα λειτουργίας, την καθυστέρηση πρόσβασης στην μνήμη και την καταναλισκόμενη ισχύ. Αυτοί οι παράγοντες συχνά αναφέρονται και ως \textsl{frequency wall, memory wall} και \textsl{power wall}. Η αρχιτεκτονική του \textsl{Cell} είναι τέτοια ώστε να αντιμετωπίζει επιτυχώς αυτούς του παράγοντες.\newline \indent

\subsubsection{Frequency Wall}
\label{subsubsection:subsub3111}
\indent
Οι συμβατικοί επεξεργαστές απαιτούν την αύξηση του βάθους του \textsl{pipeline} ώστε να επιτύχουν υψηλές συχνότητες λειτουργίας και να αυξήσουν την απόδοσή τους. Όμως, η τεχνική αυτή έχει φθάσει στο σημείο όπου επιπλέον αύξηση οδηγεί σε μειούμενα οφέλη (\textsl{law of diminishing effects}) - ακόμη και σε αρνητικές επιπτώσεις σε περίπτωση που η κατανάλωση ισχύος ληφθεί υπόψη.\newline \indent
Εξειδικεύοντας την λειτουργία του \ac{PPE} και του \ac{SPE} για εργασίες όπου κυριαρχούν οι εντολές ελέγχου (\textsl{control-intensive tasks}) και για εργασίες όπου κυριαρχούν οι εντολές υπολογισμού (\textsl{data-intensive tasks}) αντίστοιχα, η αρχιτεκτονική στην οποία βασίζεται ο \textsl{Cell} επιτρέπει τον αποδοτικό σχεδιασμό αυτών των στοιχείων για λειτουργία σε υψηλές συχνότητες χωρίς υπερβολική επιβάρυνση.\newline \indent
Το \ac{PPE} επιτυγχάνει υψηλή αποδοτικότητα βελτιστοποιώντας την εκτέλεση και των δύο διαθέσιμων νημάτων, αντί ενός, ενώ το \ac{SPE} επιτυγχάνει υψηλή αποδοτικότητα με χρήση του μεγάλου φακέλου καταχωρητών, ο οποίος επιτρέπει την ταυτόχρονη εκτέλεση αρκετών εντολών χωρίς την επιβάρυνση από τεχνικές όπως \textsl{register renaming} ή \textsl{out-of-order execution}. Επίσης, η χρήση των ασύγχρονων εντολών \ac{DMA} επιτρέπει την πρόσβαση στην μνήμη πολλαπλών εντολών χωρίς την επιβάρυνση από τον μηχανισμό του \textsl{speculation} που υπάρχει στους συμβατικούς, σύγχρονους επεξεργαστές.

\subsubsection{Memory Wall}
\label{subsubsection:subsub3112}
\indent
Οι σύγχρονοι πολυ-επεξεργαστές χαρακτηρίζονται από την μεγάλη καθυστέρηση για την πρόσβαση στην κύρια μνήμη, η οποία προσεγγίζει την τιμή των χιλίων κύκλων μηχανής. Αυτό έχει ως αποτέλεσμα, η απόδοση των εφαρμογών να εξαρτάται από την μεταφορά δεδομένων από και προς την κύρια μνήμη. Η αντιδραστική, \textsl{reactive}, φύση της κρυφής μνήμης δεν επιφέρει κάποια βελτίωση και πλέον ο μεταγλωττιστής, ή ακόμη και ο προγραμματιστής της εφαρμογής, επιφορτίζεται με την ρητή μεταφορά των δεδομένων.\newline \indent
Για την αντιμετώπιση αυτής της καθυστέρησης, τα \acp{SPE} του επεξεργαστή \textsl{Cell} υιοθετούν δύο τεχνικές:

\begin{itemize}

\item{Ιεραρχία μνήμης τριών επιπέδων (φάκελος καταχωρητών, τοπική μνήμη, κύρια μνήμη)}

\item{Ασύγχρονες εντολές \ac{DMA}}

\end{itemize}
\indent
Αυτοί οι μηχανισμοί επιτρέπουν στον προγραμματιστή της εφαρμογής να επικαλύψει την καθυστέρηση πρόσβασης στην κύρια μνήμη με την ταυτόχρονη δρομολόγηση υπολογισμών και μεταφοράς δεδομένων. Ενδεικτικά αναφέρεται ότι είναι δυνατή η υποστήριξη 128 ταυτόχρονων μεταφορών, αριθμός ο οποίος υπερβαίνει τον αριθμό των ταυτόχρονων μεταφορών που υποστηρίζονται από τους συμβατικούς επεξεργαστές κατά έναν παράγοντα ίσο με είκοσι.

\subsubsection{Power-limitation Wall}
\label{subsubsection:subsub3113}
\indent
Η απόδοση των μικροεπεξεργαστών εξαρτάται σε μεγάλο βαθμό από την κατανάλωση ισχύος και όχι από τους διαθέσιμους πόρους, όπως \textsl{transistors} και καλωδιώσεις. Επομένως, ο μόνος τρόπος για την αύξηση της απόδοσης είναι η αύξηση της αποδοτικότητας στην κατανάλωση ισχύος με την ταυτόχρονη αύξηση της συχνότητας λειτουργίας.\newline \indent
Μία μέθοδος για την επίτευξη αποδοτικότητας στην κατανάλωση ισχύος είναι ο διαχωρισμός μεταξύ επεξεργαστών που είναι βελτιστοποιημένοι για την εκτέλεση \textsl{control-intensive tasks} και επεξεργαστών που είναι βελτιστοποιημένοι για την εκτέλεση \textsl{data-intensive tasks}.\newline \indent
Όπως αναφέρθηκε και παραπάνω, ο επεξεργαστής \textsl{Cell} διαθέτει έναν επεξεργαστή, το \ac{PPE}, βελτιστοποιημένο για την εκτέλεση του λειτουργικού συστήματος και την εκτέλεση \textsl{control-intensive} λειτουργιών και οχτώ επεξεργαστές, τα \acp{SPE}, για την εκτέλεση \textsl{data-intensive} λειτουργιών. Η αρχιτεκτονική των \acp{SPE} είναι αρκετά απλή καθώς δεν υπάρχουν μηχανισμοί όπως \textsl{branch prediction, out-of-order execution, speculative execution} και \textsl{register renaming}, οι οποίοι υπάρχουν σε άλλους σύγχρονους επεξεργαστές και συμβάλλουν στην επιτάχυνση της εκτέλεσης των προγραμμάτων. Επομένως, ο αριθμός των \textsl{transistors} που εξοικονομείται από την απουσία αυτών αφιερώνεται για την εκτέλεση υπολογισμών καθιστώντας τα \acp{SPE} πολύ αποδοτικά.

\section{Η Αρχιτεκτονική του Επεξεργαστή Cell}
\label{section:sect32}
\indent
Το πρώτο πρωτότυπο του \textsl{Cell} ήταν υλοποιημένο σε τεχνολογία \textsl{90nm SOI}, αποτελούνταν από 234 εκατομμύρια \textsl{transistors}, είχε κατανάλωση ίση με 60-80 \textsl{W} και τάση τροφοδοσίας ίση με 1.1 \textsl{V} ενώ λειτουργούσε σε συχνότητα ίση με \textsl{3.2 GHz}. Η υλοποίηση του πρωτοτύπου παρουσιάζεται στο σχήμα~\ref{figure:fig32} \cite{CellProject} ενώ ένα ενδεικτικό σχήμα για την δομή του \textsl{Cell} παρουσιάζεται στο σχήμα~\ref{figure:fig33}.
\begin{figure}[b]
\centering
\includegraphics[width=2in, height=3in]{Chapter3/figures/figure2.eps}
\caption{Εικόνα από ένα chip του επεξεργαστή \textsl{Cell}.}
\label{figure:fig32}
\end{figure}

\begin{figure}[b]
\centering
\includegraphics[width=5in, height=3in]{Chapter3/figures/figure3.eps}
\caption{Δομικό διάγραμμα του επεξεργαστή \textsl{Cell}.}
\label{figure:fig33}
\end{figure}

\subsection[3.2.1 Power Processing Element (PPE)]{Power Processing Element (PPE)}
\label{subsection:sub321}
\indent
Το \acf{PPE}, όπως υποδηλώνει και το όνομα αυτού, αποτελεί έναν επεξεργαστή \ac{RISC} που υποστηρίζει την \textsl{64-bit} αρχιτεκτονική \textsl{PowerPC} με τις \ac{SIMD} επεκτάσεις αυτές, γνωστές και ως επεκτάσεις \textsl{AltiVec}. Το \ac{PPE} υποστηρίζει την ταυτόχρονη εκτέλεση δύο νημάτων, επιτρέποντας έτσι την ταυτόχρονη διεκπεραίωση δύο λειτουργιών. Στον επεξεργαστή \textsl{Cell}, το \ac{PPE} εκτελεί το λειτουργικό σύστημα και, στην πλειονότητα των περιπτώσεων, εκτελεί και τον κώδικα που ελέγχει και συντονίζει την λειτουργία των \acp{SPE}. Το σχήμα~\ref{figure:fig34} παρουσιάζει την αρχιτεκτονική του \ac{PPE}.

%\begin{figure}[t]
%\centering
%\includegraphics[width=5in, height=3in]{Chapter3/figures/figure4.eps}
%\caption{Δομικό διάγραμμα του \textsl{PPE}.}
%\label{figure:fig34}
%\end{figure}
\indent
Οι κύριες υπομονάδες που αποτελούν το \ac{PPE} είναι η μονάδα \acf{PPU} και το υποσύστημα \acf{PPSS}, το οποίο αποτελεί το υποσύστημα για την διαχείριση της μνήμης. Η μονάδα \ac{PPU} αποτελείται από τις ακόλουθες επιμέρους μονάδες:

\begin{itemize}

\item{\acf{IU}: Αυτή η υπομονάδα εκτελεί την προσκόμιση και την αποκωδικοποίηση των εντολών. Έπειτα, κατευθύνει αυτές στην κατάλληλη υπομονάδα για την εκτέλεση της εντολής. Εμπεριέχει την υπομονάδα \acf{BRU} και μία κρυφή μνήμη εντολών (\textsl{L1 instruction cache}) μεγέθους \(32KB\).}

\item{\acf{LSU}: Η υπομονάδα \ac{LSU} λαμβάνει αιτήσεις από την μνήμη, τις οποίες προωθεί στο υποσύστημα \ac{PPSS} και εμπεριέχει μία κρυφή μνήμη δεδομένων (\textsl{L1 data cache}) μεγέθους \(32KB\).}

\item{\acf{VSU}: Αυτή η υπομονάδα περιλαμβάνει την μονάδα \acf{FPU}, η οποία εκτελεί πράξεις σε βαθμωτές μεταβλητές ακεραίων κινητής υποδιαστολής, και την μονάδα \acf{VXU}, η οποία εκτελεί πράξεις σε διανύσματα (\textsl{vectors}) τύπου ακεραίων κινητής υποδιαστολής. Τα διανύσματα είναι μεγέθους \(128-bits\) και σε αυτή την μονάδα εκτελούνται και οι πράξεις που ορίζονται από τις επεκτάσεις \textsl{Vector/SIMD Multimedia Extensions (AltiVec)}.}

\item{\acf{FXU}: Η υπομονάδα \ac{FXU} εκτελεί πράξεις σε ακεραίους όπως λογικές ή αριθμητικές πράξεις.}

\item{\acf{MMU}: Η υπομονάδα \ac{MMU} διαχειρίζεται το σύστημα εικονικής μνήμης του \textsl{Cell} και είναι υπεύθυνη για την μετάφραση των \textsl{ενεργών διευθύνσεων} (\textsl{effective addresses}) σε \textsl{εικονικές} και \textsl{πραγματικές διευθύνσεις}.}

\item{\acf{BRU}: Η υπομονάδα \ac{BRU} επεξεργάζεται τις εντολές διακλαδώσεων που υπάρχουν στο κώδικα της εκάστοτε εφαρμογής.}

\end{itemize}
\indent
Η ιεραρχία κρυφής μνήμης του \ac{PPE} αποτελείται από μία \textsl{Level-1} κρυφή μνήμη εντολών, με μέγεθος ίσο με \(32\ KB\), μία \textsl{Level-1} κρυφή μνήμη δεδομένων με μέγεθος ίσο με \(32\ KB\), και μία ενοποιημένη \textsl{Level-2 write-back} μνήμη εντολών και δεδομένων με μέγεθος ίσο με \(512\ KB\). Οι δύο \textsl{Level-1} μνήμες υποστηρίζουν συνεκτική πρόσβαση στα δεδομένα, με κάθε φόρτωση να έχει μέγεθος ίσο με \(16\) ή \(32\) bytes.\newline \indent
Η αρχιτεκτονική του \ac{PPE} υποστηρίζει το σύνολο εντολών της αρχιτεκτονικής \textsl{PPC 970} και το πλήρες σύνολο των εντολών \textsl{AltiVec/VMX}. Όπως κάθε αρχιτεκτονική τύπου \textsl{RISC}, η πλειονότητα των εντολών εκτελούν πράξεις μεταξύ καταχωρητών ενώ συγκεκριμένες εντολές χρησιμοποιούνται για πρόσβαση στην μνήμη.\newline \indent
Επίσης, υποστηρίζεται ένα μικρό σύνολο διαφορετικών κωδικοποιήσεων για τις διαθέσιμες εντολές ενώ όλες οι διαφορετικές κωδικοποιήσεις έχουν το ίδιο μέγεθος. Το σύνολο καταχωρητών που είναι διαθέσιμο στον προγραμματιστή αποτελείται από 32 καταχωρητές γενικού σκοπού με μέγεθος ίσο με \(64\ bits\), 32 καταχωρητές με μέγεθος ίσο με \(64\ bits\) για την αποθήκευση αριθμών κινητής υποδιαστολής που ακολουθούν το πρότυπο \textsl{IEEE-754} και 32 διανυσματικούς καταχωρητές μεγέθους ίσο με \(128\ bits\) για την αποθήκευση δεδομένων που χρησιμοποιούνται σε διανυσματικές πράξεις.\newline \indent
Όπως αναφέρθηκε παραπάνω και παρουσιάζεται και στο σχήμα~\ref{figure:fig34}, το \ac{PPE} αποτελεί έναν επεξεργαστή \textsl{RISC}, με δύο \textsl{hardware threads} ενώ δεν υποστηρίζεται η εκτέλεση εντολών εκτός σειράς. Σε κάθε κύκλο ρολογιού εισέρχονται προς επεξεργασία δύο εντολές, οι οποίες κατευθύνονται προς την κατάλληλη υπομονάδα εκτέλεσης όπως αυτή καθορίζεται από την υπομονάδα \ac{IU}. Σε περίπτωση που δεν υπάρχουν εξαρτήσεις μεταξύ των εντολών, αυτές εκτελούνται παράλληλα οπότε αυξάνεται η ικανότητα διεκπεραίωσης του επεξεργαστή. Επίσης, κατά την φάση της αποκωδικοποίησης ελέγχεται κατά πόσο η τρέχουσα εντολή αποτελεί μία \textsl{microcoded} εντολή. Η εκτέλεση τέτοιου είδους εντολών απαιτεί τον διαχωρισμό αυτών σε επιμέρους, απλούστερες εντολές. Γι' αυτό το στάδιο απαιτούνται 11 κύκλοι ρολογιού οπότε οι \textsl{microcoded} εντολές θα πρέπει να αποφεύγονται λόγω της μεγάλης καθυστέρησης που εισάγουν.\newline \indent
Τέλος, ένα σημαντικό τμήμα της αρχιτεκτονικής κάθε επεξεργαστή είναι η πρόβλεψη διακλαδώσεων (\textsl{branch prediction}) λόγω της μεγάλης καθυστέρησης που εισάγει μία λανθασμένη πρόβλεψη, η οποία ανέρχεται σε 23 κύκλους ρολογιού λόγω της επιβάρυνσης που εισάγεται από την διαδικασία αδειάσματος (\textsl{flushing}) του \textsl{pipeline}. Το \ac{PPE} υλοποιεί την πρόβλεψη διακλαδώσεων όπως ορίζεται από την αρχιτεκτονική \textsl{PPC 970}. Η αρχιτεκτονική συνόλου εντολών του \textsl{PowerPC} εμπεριέχει μία πληθώρα εντολών διακλάδωσης, πολλές από τις οποίες χρησιμοποιούνται ώστε να παρέχουν υποδείξεις στην υπομονάδα \ac{IU} ως προς το ποιο είναι το πιο πιθανό αποτέλεσμα της διακλάδωσης. Ένα χαρακτηριστικό παράδειγμα εντολής υψηλού επιπέδου είναι η εντολή \textsl{\_\_builtin\_expect}. Αυτού του είδους οι εντολές αναγνωρίζονται από τους αντίστοιχους μεταγλωττιστές οι οποίοι εισάγουν τις κατάλληλες εντολές \textsl{assembly}.
\begin{figure}[t]
\centering
\includegraphics[width=5in, height=3in]{Chapter3/figures/figure4.eps}
\caption{Δομικό διάγραμμα του \textsl{PPE}.}
\label{figure:fig34}
\end{figure}
\newline \indent
Όσον αφορά στον προγραμματισμό του \ac{PPE}, αυτός πραγματοποιείται χρησιμοποιώντας τις υψηλού επιπέδου γλώσσες \textsl{C/C++} ενώ παρέχεται και ένα σύνολο εσωτερικών εντολών\footnote{\small Οι εντολές αυτές ονομάζονται \textsl{intrinsics} και αντιστοιχούν σε μία ή περισσότερες εντολές \textsl{assembly}.}, όπως για παράδειγμα αυτές που υλοποιούν τις επεκτάσεις \textsl{AltiVec/VMX}.

\subsection[3.2.2 Synergistic Processing Element (SPE)]{Synergistic Processing Element (SPE)}
\label{subsection:sub322}
\indent
Παρόλη την υπολογιστική δύναμη του \ac{PPE}, η ασυνήθης υπολογιστική δύναμη του επεξεργαστή \textsl{Cell} πηγάζει από το \acf{SPE}. Σε αντίθεση με την αρχιτεκτονική γενικού σκοπού του \ac{PPE}, η αρχιτεκτονική του \ac{SPE} σχεδιάστηκε με κύριο γνώμονα την εκτέλεση διανυσματικών υπολογισμών με πολύ μεγάλη ταχύτητα. Το \textsl{datapath} της αρχιτεκτονικής έχει εύρος ίσο με \(128\ bits\) ενώ το \textsl{pipeline} είναι \textsl{dual-issue}, δηλαδή υποστηρίζεται η εκτέλεση δύο εντολών σε κάθε κύκλο ρολογιού. Όμως, η εκτέλεση δύο εντολών είναι δυνατή μόνο στην περίπτωση που δεν υπάρχουν εξαρτήσεις μεταξύ αυτών ενώ σε κάθε \textsl{pipeline} εκτελούνται εντολές από συγκεκριμένες κλάσεις εντολών. Καθένα από τα οχτώ \acp{SPE} αποτελείται από την υπομονάδα \ac{SPU} και τον \textsl{ελεγκτή ροής μνήμης} (\ac{MFC}). Το δομικό διάγραμμα του \ac{SPE} όπως επίσης και τα στάδια του \textsl{pipeline} για κάθε μία από τις διαθέσιμες κλάσεις εντολών παρουσιάζονται στο σχήμα~\ref{figure:fig35}.

\begin{figure}
\centering
\includegraphics[width=5in, height=4in]{Chapter3/figures/figure5.eps}
\caption{Δομικό διάγραμμα του \textsl{SPE}.}
\label{figure:fig35}
\end{figure}
\indent
Οι κύριες υπομονάδες που αποτελούν κάθε \ac{SPE} είναι οι ακόλουθες:

\begin{itemize}

\item{SPU Control Unit (SCN): Η μονάδα αυτή προσκομίζει και διανέμει τις εντολές στις αντίστοιχες μονάδες εκτέλεσης εντολών. Επίσης εκτελεί τις διάφορες εντολές ελέγχου, όπως για παράδειγμα εντολές διακλαδώσεων.}

\item{SPU Even Fixed-Point Unit (SFX): Σε αυτή την μονάδα εκτελούνται λογικές και αριθμητικές εντολές, εντολές σύγκρισης όπως επίσης και εντολές για αντιστροφή αριθμών κινητής υποδιαστολής.}

\item{SPU Odd Fixed-Point Unit (SFS): Η μονάδα αυτή είναι υπεύθυνη για την εκτέλεση πράξεων ολίσθησης, περιστροφής και ανακατέματος (\textsl{shuffling}).}

\item{SPU Floating-Point Unit (SFP): Στην εν λόγω μονάδα εκτελούνται οι αριθμητικές πράξεις σε ακεραίους κινητής υποδιαστολής, απλής ή διπλής ακρίβειας, πράξεις πολλαπλασιασμού ακεραίων όπως επίσης και διάφορες πράξεις μετατροπής ακεραίων.}

\item{SPU Load/Store Unit (SLS): Η μονάδα \textsl{SLS} είναι υπεύθυνη για την διαχείριση των διαφόρων εντολών που αφορούν στην μνήμη. Εδώ εκτελούνται εντολές φόρτωσης και αποθήκευσης στην τοπική μνήμη όπως επίσης και εντολές για αιτήσεις \ac{DMA} στην τοπική μνήμη.}

\item{SPU Channel and DMA Unit (SSC): Αυτή η μονάδα εκτελεί τις διάφορες λειτουργίες επικοινωνίας με τον \textsl{ελεγκτή ροής μνήμης} (\ac{MFC}) και διαχειρίζεται τις μεταφορές δεδομένων μέσω του μηχανισμού \ac{DMA}.}

\end{itemize}
\begin{table}
\centering
\begin{tabular}{|c|c|c|}
  \hline
  Τύπος Δεδομένων & Περιεχόμενα Τύπου Δεδομένων & Μέγεθος (bytes) \\ \hline
  boolean & True/False & 1 \\ \hline
  char & Χαρακτήρας ASCII & 1 \\ \hline
  unsigned char & Ακέραιος (0-255) & 1 \\ \hline
  short & Προσημασμένος ακέραιος τύπου short & 2 \\ \hline
  unsigned short & Μη προσημασμένος ακέραιος τύπου short & 2 \\ \hline
  int & Προσημασμένος Ακέραιος & 4 \\ \hline
  unsigned int & Μη προσημασμένος ακέραιος & 4 \\ \hline
  float & Ακέραιος κινητής υποδιαστολής, απλής ακρίβειας & 4 \\ \hline
  double & Ακέραιος κινητής υποδιαστολής, διπλής ακρίβειας & 8 \\ \hline
  long long / long long int & Προσημασμένος ακέραιος τύπου long long & 8 \\ \hline
  unsigned long long / & & \\
  unsigned long long int & Μη προσημασμένος ακέραιος τύπου long long & 8 \\ \hline
  qword & Εξαρτώνται από την εντολή & 16 \\
  \hline
\end{tabular}
\caption{Τύποι δεδομένων βαθμωτών μεταβλητών στα \textsl{SPUs}}
\label{table:tab31}
\end{table}
\indent
Τα χαρακτηριστικά που διαφοροποιούν την αρχιτεκτονική του \ac{SPE} από την αρχιτεκτονική του \ac{PPE} είναι η απουσία κρυφής μνήμης (\textsl{cache hierarchy}) και εικονικής μνήμης, η απουσία μονάδων που επεξεργάζονται βαθμωτές μεταβλητές, καθώς όλες οι πράξεις πραγματοποιούνται σε διανύσματα, και τα δύο διαφορετικά \textsl{pipelines}.\newline \indent
Κάθε \ac{SPE} αποτελείται από μία τοπική μνήμη, αποκλειστικής πρόσβασης και μεγέθους \(256\ KB\) (\ac{LS}), η οποία είναι οργανωμένη σε τέσσερεις \textsl{SRAM arrays} μεγέθους \(64\ KB\). Η μνήμη έχει μόνο μία θύρα ανάγνωσης/εγγραφής και γι' αυτό οι λειτουργίες πρόσβασης σε αυτή είναι \textsl{fully pipelined} ώστε να επιτυγχάνεται ταχύτερη πρόσβαση σε αυτή. Το εύρος ζώνης της μνήμης είναι ίσο με \(16\ bytes\) ανά κύκλο ενώ εντολές τύπου \textsl{DMA} μεταφέρουν από και προς την μνήμη δεδομένα που είναι οργανωμένα σε μονάδες μεγέθους \(1024\ bytes\).\newline \indent
Ένα ιδιαίτερο χαρακτηριστικό της μνήμης είναι ότι η πρόσβαση σε αυτή ξεκινά από διευθύνσεις που είναι πολλαπλάσιες των \(16\ bytes\) (quadword alignment). Σε περίπτωση που αυτό δεν ισχύει, εκτελούνται επιπλέον πράξεις για την ικανοποίηση των περιορισμών ευθυγράμμισης. Όπως είναι εμφανές, για την ικανοποίηση των περιορισμών πρέπει να ληφθεί ειδική μέριμνα από τον προγραμματιστή της εφαρμογής προκειμένου να μην επιβαρύνεται η εκτέλεση της εφαρμογής με το κόστος που συνεπάγονται οι προαναφερθείσες πράξεις ευθυγράμμισης. \newline \indent
Καθώς η μνήμη έχει μόνο μία θύρα εγγραφής/ανάγνωσης, όλες οι εντολές προσπέλασης στην μνήμη, όπως εντολές φόρτωσης και αποθήκευσης, συναγωνίζονται για την πρόσβαση σε αυτή την θύρα. Για την επίλυση αυτών των συγκρούσεων, επιβάλλεται ένα σχήμα προτεραιοτήτων μεταξύ αυτών των εντολών. Οι εντολές του μηχανισμού \ac{DMA} έχουν την μεγαλύτερη προτεραιότητα, μετά ακολουθούν οι εντολές φόρτωσης και αποθήκευσης ενώ την χαμηλότερη προτεραιότητα έχουν οι προσπελάσεις για την προσκόμιση εντολών.\newline \indent
Ο φάκελος καταχωρητών σε κάθε \ac{SPE} έχει μέγεθος ίσο με 128 καταχωρητές, όπου κάθε καταχωρητής έχει μήκος ίσο με \(128\ bits\). Οι διαθέσιμοι τύποι δεδομένων για τις βαθμωτές μεταβλητές και τις διανυσματικές μεταβλητές σε κάθε \ac{SPU}, όπως επίσης και το μέγεθος αυτών αναφέρονται στους πίνακες~\ref{table:tab31} και~\ref{table:tab32}.
\begin{table}
\centering
\begin{tabular}{|c|c|}
  \hline
  Διανυσματικός Τύπος Δεδομένων & Πλήθος Στοιχείων του Καταχωρητή \\ \hline
  vector unsigned char & 16 μη προσημασμένοι χαρακτήρες, \\
                       & μεγέθους 8-bit \\ \hline
  vector signed char & 16 προσημασμένοι χαρακτήρες, \\
                     & μεγέθους 8-bit \\ \hline
  vector unsigned short & 8 μη προσημασμένοι ακέραιοι τύπου short, \\
                        & μεγέθους 16-bit \\ \hline
  vector signed short & 8 προσημασμένοι ακέραιοι τύπου short, \\
                      & μεγέθους 16-bit \\ \hline
  vector pixel & 8 μη προσημασμένες μεταβλητές τύπου halfword, \\
               & μεγέθους 16-bit \\ \hline
  vector unsigned int & 4 μη προσημασμένοι ακέραιοι \\ \hline
  vector signed int & 4 προσημασμένοι ακέραιοι \\ \hline
  vector float & 4 μεταβλητές κινητής υποδιαστολής, απλής ακρίβειας \\ \hline
  vector unsigned long long & 2 μη προσημασμένες μεταβλητές τύπου long long, \\
                            & μεγέθους 64-bit \\ \hline
  vector signed long long &  2 προσημασμένες μεταβλητές τύπου long long, \\
                          & μεγέθους 64-bit \\ \hline
  vector double & 2 μεταβλητές κινητής υποδιαστολής, διπλής ακρίβειας \\ \hline
\end{tabular}
\caption{Τύποι δεδομένων διανυσμάτων στα \textsl{SPUs}}
\label{table:tab32}
\end{table}
Η αποθήκευση βαθμωτών μεταβλητών στους διαθέσιμους καταχωρητές δίνει την δυνατότητα για εκτέλεση πράξεων μεταξύ βαθμωτών μεταβλητών, παρόλο που τα \acp{SPE} έχουν σχεδιαστεί κυρίως για πράξεις μεταξύ διανυσμάτων. Για την εκτέλεση αυτών των πράξεων, οι βαθμωτές μεταβλητές πρέπει να αποθηκευθούν σε μία προκαθορισμένη θέση του αντίστοιχου καταχωρητή, η οποία ονομάζεται \textsl{preferred slot}.\newline \indent
Η εικόνα~\ref{figure:fig36} παρουσιάζει αυτή την θέση για τους διαθέσιμους τύπους δεδομένων. Επί παραδείγματι, μία βαθμωτή μεταβλητή μήκους ίσου με 2 bytes αποθηκεύεται στις θέσεις 2 έως 3 του καταχωρητή\footnote{\small Η διευθυνσιοδότηση των θέσεων των καταχωρητών γίνεται βάσει \textsl{big-endian} διάταξης}. Αυτός ο περιορισμός συνεπάγεται και επιπλέον κόστος σε περίπτωση που εκτελούνται πράξεις μεταξύ βαθμωτών μεταβλητών. Καθώς η φόρτωση από την μνήμη γίνεται σε μονάδες των 128 bits, για την αποθήκευση των βαθμωτών μεταβλητών στο \textsl{preferred slot}, ενδέχεται να απαιτούνται επιπλέον πράξεις ολισθήσεις των περιεχομένων των καταχωρητών. Αυτές οι πράξεις δημιουργούν εξαρτήσεις εντολών που έχουν ως αποτέλεσμα την ύπαρξη παγωμάτων (\textsl{stalls}) στο \textsl{pipeline}, τα οποία οδηγούν σε μείωση της επιτεύξιμης απόδοσης και αύξηση του χρόνου εκτέλεσης του προγράμματος.\newline \indent
Όσον αφορά στην αρχιτεκτονική του \textsl{pipeline} του \ac{SPU}, όπως αναφέρθηκε και παραπάνω, αυτή αποτελείται από δύο \textsl{pipelines}, τα οποία χωρίζονται επιμέρους σε άρτιο και περιττό \textsl{pipeline}. Κατά κύριο λόγο, στο άρτιο \textsl{pipeline} εκτελούνται εντολές ακεραίων σταθερής και κινητής υποδιαστολής ενώ στο περιττό \textsl{pipeline} εκτελούνται εντολές φόρτωσης από την μνήμη όπως επίσης και εντολές μετάθεσης περιεχομένων της μνήμης. Με χρήση αυτού του μηχανισμού, καθίσταται δυνατή η εκτέλεση δύο εντολών σε κάθε κύκλο μηχανής, υπό την προϋπόθεση ότι υπάρχουν αρκετές διαθέσιμες εντολές και δεν υπάρχουν εξαρτήσεις μεταξύ των εντολών. Κατ' αυτό τον τρόπο, είναι δυνατή η εκμετάλλευση του εγγενούς παραλληλισμού μεταξύ των εντολών (\ac{ILP}) που υπάρχει στις εφαρμογές. \newline \indent
Η αρχιτεκτονική των \acp{SPE}, με τα προαναφερθέντα χαρακτηριστικά, επιτρέπει την εκτέλεση \(6.4\) \acf{GFlops} για πράξεις με αριθμούς κινητής υποδιαστολής διπλής ακρίβειας και \(25.6\) \ac{GFlops} για πράξεις με αριθμούς κινητής υποδιαστολής απλής ακρίβειας. Όπως παρατηρούμε, η εκτέλεση πράξεων κινητής υποδιαστολής διπλής ακρίβειας δεν είναι τόσο αποδοτική. Αυτό οφείλεται στο ότι οι εν λόγω πράξεις δεν είναι πλήρως \textsl{pipelined} ενώ κατά την εκτέλεση τέτοιων πράξεων δεν είναι δυνατή η χρήση του \textsl{dual-issue} από άλλες εντολές. Η πιο πρόσφατη υλοποίηση της αρχιτεκτονικής \ac{CBEA}, ο επεξεργαστής \textsl{PowerXCell 8i}\footnote{\small Ο επεξεργαστής PowerXCell 8i είναι αυτός που χρησιμοποιείται στα συστήματα \textsl{QS22 Blades} της \textsl{IBM}} \cite{PowerXCell}, ενσωματώνει μία βελτιστοποιημένη μονάδα εκτέλεσης υπολογισμών κινητής υποδιαστολής διπλής ακρίβειας στα \acp{SPE}, με αποτέλεσμα την αύξηση της υπολογιστικής ικανότητας του κάθε \ac{SPE} σε \(12.8\) \ac{GFlops}.

\begin{figure}
\centering
\includegraphics[width=5in, height=3in]{Chapter3/figures/figure6.eps}
\caption{Αποθήκευση βαθμωτών μεταβλητών στους καταχωρητές.}
\label{figure:fig36}
\end{figure}

\indent
Ο ελεγκτής ροής μνήμης (\ac{MFC}) αποτελεί το μέσο για την επικοινωνία του \ac{SPU} με την εξωτερική μνήμη, άλλα επεξεργαστικά στοιχεία όπως επίσης και συσκευές εισόδου/εξόδου. Για την επικοινωνία χρησιμοποιείται το \acf{EIB}, το οποίο περιγράφεται στην ενότητα~\ref{subsection:sub323}. Ο ελεγκτής υποστηρίζει τον μηχανισμό \ac{DMA}, μέσω του οποίου καθίσταται δυνατή η μεταφορά μεγάλου όγκου δεδομένων και εντολών από και προς την τοπική μνήμη. Εκτός του μηχανισμού \ac{DMA}, ο ελεγκτής ροής μνήμης υποστηρίζει τον μηχανισμό των σημάτων (\textsl{signals}) και των γραμματοκιβωτίων (\textsl{mailboxes}), οι οποίοι είναι δύο επιπλέον μηχανισμοί που προσφέρονται για την επικοινωνία μεταξύ των επεξεργαστικών στοιχείων (\textsl{inter-processor communication}). Αυτοί οι μηχανισμοί χρησιμοποιούνται για την ανταλλαγή μηνυμάτων μικρού μεγέθους μεταξύ των επεξεργαστών, όπως για παράδειγμα σήματα περάτωσης εργασίας ή διευθύνσεις δεδομένων.

\subsection[3.2.3 Element Interconnect Bus (EIB)]{Element Interconnect Bus (EIB)}
\label{subsection:sub323}
\indent
Το \acf{EIB} είναι ένας κυκλικός δίαυλος, ο οποίος αποτελείται από τέσσερεις επιμέρους δακτυλίους με εύρος ίσο με \(16\ bytes\), όσο είναι και το μήκος μίας γραμμής στην \ac{LS} των \acp{SPE}, και έχει συχνότητα λειτουργίας ίση με το μισό της συχνότητας λειτουργίας του επεξεργαστή. Οι δύο δακτύλιοι μεταφέρουν δεδομένα με φορά αυτή των δεικτών του ρολογιού ενώ οι άλλοι δύο μεταφέρουν δεδομένα με αντίθετη φορά. Το \ac{EIB} χρησιμοποιείται για την μεταφορά δεδομένων από το \ac{PPE} και την \textsl{L2 cache} αυτού προς τα \acp{SPE} και αντίστροφα. Επίσης, συνδέεται με την διεπαφή του ελεγκτή μνήμης (\textsl{Memory Interface Controller}) και με το \textsl{FlexIO} για εξωτερική επικοινωνία. Το ακόλουθο σχήμα παρουσιάζει την αρχιτεκτονική του \ac{EIB}.

\begin{figure}
\centering
\includegraphics[width=5in, height=3in]{Chapter3/figures/figure7.eps}
\caption{Δομικό διάγραμμα του \textsl{EIB}.}
\label{figure:fig37}
\end{figure}

\indent
Κάθε επεξεργαστής του συστήματος υποστηρίζει την ταυτόχρονη λήψη και αποστολή δεδομένων μέσω του \ac{EIB}.
Το μέγιστο εύρος ζώνης που υποστηρίζεται είναι ίσο με \(96\ bytes\) και κάθε δακτύλιος μπορεί να υποστηρίξει ταυτόχρονα μέχρι τρεις αιτήσεις \ac{DMA} υπό την προϋπόθεση ότι δεν υπάρχει επικάλυψη μεταξύ αυτών. Τέλος, ο μέγιστος αριθμός των εκκρεμών αιτήσεων που υποστηρίζεται είναι ίσος με \(128\) αιτήσεις για μεταφορά δεδομένων από την κύρια μνήμη προς τα \acp{SPE} και αντίστροφα.

\subsection[3.2.4 Μνήμη και Είσοδος/Έξοδος Δεδομένων]{Μνήμη και Είσοδος/Έξοδος Δεδομένων}
\label{subsection:sub324}
Ο ελεγκτής διεπαφής μνήμης (\textsl{MIC}) του επεξεργαστή \textsl{Cell} υποστηρίζει μία ή δύο μνήμες υψηλής ταχύτητας τύπου \acf{XDR}. Το εύρος κάθε καναλιού ισούται με \(12.8\ GB/s\) οπότε το μέγιστο εύρος ζώνης που δύναται να υποστηριχθεί είναι ίσο με \(25.6\ GB/s\). Η συνολική μνήμη που μπορεί να υποστηριχθεί, εξαρτάται από την διαμόρφωση του συστήματος, και κυμαίνεται από \(64\ MB\) έως \(64\ GB\) μνήμης \textsl{XDR DRAM}.\newline \indent
Η μονάδα \acf{BEI} του επεξεργαστή \textsl{Cell} είναι μία διεπαφή που χρησιμοποιείται από τους επεξεργαστές του συστήματος για την επικοινωνία με τις συσκευές \textsl{Εισόδου-Εξόδου} του συστήματος. Αυτή αποτελείται από την μονάδα \acf{BIC}, την μονάδα \acf{IOC} και την μονάδα \acf{IIC}. Η μονάδα \ac{BEI} υποστηρίζει δύο διεπαφές \textsl{Rambus FlexIO}. Η πρώτη διεπαφή αποτελεί μία μη συνεκτική διεπαφή για \textsl{Είσοδο/Έξοδο} και μπορεί να χρησιμοποιηθεί για την σύνδεση συσκευών όπως κάρτες γραφικών ή κάρτες ήχου. Η δεύτερη διεπαφή μπορεί να υποστηρίξει τόσο συνεκτική όσο και μη συνεκτική ανταλλαγή δεδομένων και μπορεί να χρησιμοποιηθεί για την επέκταση του \ac{BEI} και την σύνδεση με έναν δεύτερο επεξεργαστή \textsl{Cell BE}, ώστε να είναι δυνατή η αποστολή και λήψη δεδομένων μεταξύ αυτών. Το εύρος ζώνης που υποστηρίζεται ισούται με \(76.8\ GB/s\).

\section{Ανάπτυξη Εφαρμογών στον Επεξεργαστή Cell}
\label{section:sect33}
\indent
Η ιδιαίτερη αρχιτεκτονική του επεξεργαστή \textsl{Cell BE}, με την ύπαρξη ετερογενών επεξεργαστικών στοιχείων, εισάγει επιπλέον πολυπλοκότητα στην ανάπτυξη των εφαρμογών. Ήδη υπάρχοντα προγραμματιστικά μοντέλα χαμηλού επιπέδου μπορούν να χρησιμοποιηθούν αλλά θα πρέπει να τροποποιηθούν, ώστε να είναι δυνατή η εκτέλεση προγραμμάτων που κάνουν χρήση αυτών των μοντέλων στον επεξεργαστή \textsl{Cell}. Για την ανάπτυξη εφαρμογών για τον επεξεργαστή \textsl{Cell} διατίθεται από την \textsl{IBM} το περιβάλλον ανάπτυξης \textsl{Cell BE \ac{SDK}}. Το περιβάλλον επιτρέπει την ανάπτυξη εφαρμογών για τον επεξεργαστή σε διάφορες πλατφόρμες (\textsl{x86, x86-64, 64-bit PowerPC, Cell BE Blade Center}) και περιλαμβάνει:
\begin{itemize}

\item{Toolchains για την ανάπτυξη προγραμμάτων τόσο για το \ac{PPE} όσο και για το \ac{SPE}, για καθεμία από τις υποστηριζόμενες πλατφόρμες.}

\item{Βιβλιοθήκες και υποδείγματα κώδικα.}

\item{Έναν εξομοιωτή συστήματος που περιλαμβάνει έναν επεξεργαστή \textsl{Cell}.}

\item{Έναν πυρήνα του λειτουργικού συστήματος \textsl{Linux}.}

\item{Ένα \textsl{IDE}, το οποίο βασίζεται στην πλατφόρμα \textsl{Eclipse}, και ενοποιεί τα \textsl{toolchains}, τους μεταγλωττιστές, τον προσομοιωτή και τα άλλα εργαλεία ανάπτυξης.}

\item{Βοηθητικά Προγράμματα όπως το \textsl{spu-timing} το οποίο εκτελεί στατική ανάλυση του κώδικα, και το \textsl{FDPR-Pro}, το οποίο είναι εργαλείο για την βελτιστοποίηση ενός προγράμματος μέσω \textsl{feedback} από την εκτέλεση αυτού για έναν τυπικό φόρτο εργασίας.}

\end{itemize}

\subsection[3.3.1 Διαθέσιμοι Μεταγλωττιστές]{Διαθέσιμοι Μεταγλωττιστές}
\label{subsection:sub331}
\indent
Για την μεταγλώττιση προγραμμάτων που πρόκειται να εκτελεστούν στον επεξεργαστή \textsl{Cell} διατίθεται τόσο μία τροποποιημένη έκδοση του μεταγλωττιστή \textsl{GCC} από την \textsl{Sony} όσο και μία έκδοση του μεταγλωττιστή \textsl{XL C/C++} της \textsl{IBM}. Και οι δύο μεταγλωττιστές προορίζονται για την μεταγλώττιση προγραμμάτων για το \ac{PPE} και το \ac{SPE}.\newline \indent
Ο μεταγλωττιστής \textsl{IBM XL C/C++} έχει τροποποιηθεί ώστε να είναι δυνατή η εκμετάλλευση των δυνατοτήτων των επεξεργαστών της αρχιτεκτονικής \ac{CBE}. Αυτός ο μεταγλωττιστής προσφέρει αυξημένες δυνατότητες, τόσο για αυτόματη όσο και για κατευθυνόμενη από τον χρήστη, παραλληλοποίηση και τεμαχισμό της εφαρμογής για την εκμετάλλευση των διαφόρων ειδών παραλληλισμού που υπάρχουν στις διάφορες εφαρμογές. Οι οδηγίες, \textsl{user directives}, που χρησιμοποιούνται από τον χρήστη για την επικοινωνία με τον επεξεργαστή βασίζονται στο \textsl{OpenMP} μοντέλο προγραμματισμού. Αυτή η προσέγγιση επιτρέπει στον προγραμματιστή την θεώρηση του συστήματος ως ένα σύστημα με κοινό χώρο διευθύνσεων μνήμης (\textsl{shared-memory address space}), όπου όλα τα δεδομένα που πρόκειται να προσπελαστούν εμπεριέχονται σε αυτό τον χώρο.\newline \indent
Και οι δύο μεταγλωττιστές εμπεριέχουν ένα πλούσιο σύνολο \textsl{intrinsincs}, τα οποία μπορούν να χρησιμοποιηθούν σε συνδυασμό με τις γλώσσες προγραμματισμού \textsl{C/C++}. Αυτές οι επεκτάσεις καθιστούν τον προγραμματισμό για εκμετάλλευση των \ac{SIMD} δυνατοτήτων των \acp{SPE} και του \ac{PPE} αρκετά απλό, επιτρέποντας στον προγραμματιστή να έχει τον πλήρη έλεγχο τόσο των εντολών \ac{SIMD} όσο και της διάταξης των δεδομένων, ενώ ο μεταγλωττιστής είναι υπεύθυνος για την δρομολόγηση των εντολών και την δέσμευση των κατάλληλων καταχωρητών. Το γεγονός αυτό προσφέρει στον προγραμματιστή το πλεονέκτημα του ελέγχου των μετασχηματισμών σε υψηλό επίπεδο, όπως για παράδειγμα το ξεδίπλωμα βρόχου (\textsl{loop unrolling}), ενώ ταυτόχρονα μπορεί να εκτελέσει και βελτιστοποιήσεις χαμηλού επιπέδου.\newline \indent
Η πιο προσφιλής πρακτική για την μεταγλώττιση, διασύνδεση και εκτέλεση μίας εφαρμογής στον επεξεργαστή παρουσιάζεται στο σχήμα~\ref{figure:fig38}. Αρχικά, πραγματοποιείται η μεταγλώττιση και η διασύνδεση του εκτελέσιμου για τα \acp{SPE}. Έπειτα, το εκτελέσιμο ενσωματώνεται σε ένα \textsl{object file} το οποίο θα προσπελαστεί από το \ac{PPU}. Το παραγόμενο αρχείο βασίζεται στο format \acf{CESOF}, το οποίο είναι ένα \textsl{relocatable} αρχείο που προορίζεται για το \ac{PPU}. Τέλος, πραγματοποιείται η μεταγλώττιση του κώδικα για το \ac{PPU}, η διασύνδεση με τις όποιες απαραίτητες βιβλιοθήκες όπως επίσης και με το \textsl{object file} του κώδικα για τα \acp{SPE} και η παραγωγή του τελικού εκτελέσιμου αρχείου.

\begin{figure}
\centering
\includegraphics[width=5in, height=2.5in]{Chapter3/figures/figure8.eps}
\caption{Διαδικασία μεταγλώττισης προγραμμάτων.}
\label{figure:fig38}
\end{figure}

\subsection[3.3.2 Προγραμματισμός των SPEs]{Προγραμματισμός των SPEs}
\label{subsection:sub332}
\indent
Οι περισσότερες από τις εφαρμογές που εκτελούνται στον επεξεργαστή \textsl{Cell} χρησιμοποιούν τα διαθέσιμα \acp{SPE} στα οποία αναθέτουν επιμέρους εργασίες προς εκτέλεση. Προγραμματιστικά μοντέλα όπως το μοντέλο \textsl{Streaming} ή το μοντέλο \textsl{Pipeline} μπορούν να χρησιμοποιηθούν ανάλογα με τα χαρακτηριστικά της εφαρμογής. Ωστόσο, ανεξάρτητα από το προγραμματιστικό μοντέλο που χρησιμοποιείται, το κυρίως νήμα εκτελείται στο \ac{PPE} και είναι αυτό που δημιουργεί τα επιμέρους νήματα που θα εκτελεστούν στα \acp{SPE}. Τα προγράμματα που εκτελούνται στα \acp{SPE} είναι αυτόνομα και λαμβάνουν τις απαραίτητες παραμέτρους από το κυρίως νήμα που εκτελείται στο \ac{PPE}. Η διαχείριση των νημάτων και της μεταφοράς των δεδομένων εξαρτάται από το προγραμματιστικό μοντέλο.\newline \indent
Η διαχείριση του κώδικα που εκτελείται στα \acp{SPE} πραγματοποιείται μέσω της βιβλιοθήκης \textsl{libspe2} (SPE runtime management library) και της βιβλιοθήκης \textsl{Pthread}. Η βιβλιοθήκη \textsl{libspe2} παρέχει ένα \acf{API} χαμηλού επιπέδου το οποίο επιτρέπει στις εφαρμογές την πρόσβαση στα \acp{SPE} για την εκτέλεση κάποιων από τα νήματα της εφαρμογής. Η βιβλιοθήκη \textsl{Pthread} χρησιμοποιείται σε συνδυασμό με την βιβλιοθήκη \textsl{libspe2} για την δημιουργία των απαραίτητων νημάτων.\newline \indent
Στην γενική περίπτωση, ο έλεγχος των φυσικών πόρων των \acp{SPE} πραγματοποιείται από το λειτουργικό σύστημα και όχι από τις εφαρμογές. Οι εφαρμογές ελέγχουν και χρησιμοποιούν δομές δεδομένων σε λογισμικό οι οποίες ονομάζονται \textsl{SPE contexts}. Οι δομές αυτές αποτελούν μία λογική αναπαράσταση των \acp{SPE} και εμπεριέχουν όλη την πληροφορία που απαιτείται για την λογική περιγραφή του \ac{SPE} ενώ η επεξεργασία αυτών πραγματοποιείται από την βιβλιοθήκη \textsl{libspe2} και έτσι παρέχεται η απαραίτητη αφαίρεση στον προγραμματιστή της εφαρμογής. Το λειτουργικό σύστημα είναι υπεύθυνο για την δρομολόγηση των \textsl{contexts} από όλες τις εφαρμογές βάσει των προτεραιοτήτων και των πολιτικών δρομολόγησης.\newline \indent

\subsubsection{Εντολές SIMD}
\label{subsubsection:subsub3321}
\indent
Η εκτέλεση υπολογισμών \acf{SIMD} είναι ένα από τα πιο σημαντικά χαρακτηριστικά των \acp{SPE}. Η πλειονότητα των \textsl{compute-bound} εφαρμογών, όπως εφαρμογές πολυμέσων ή εφαρμογές επεξεργασίας εικόνας, εκτελούν τους ίδιους υπολογισμούς σε ένα μεγάλο σύνολο δεδομένων οπότε η ύπαρξη πράξεων \ac{SIMD} είναι αναγκαία για την επιτάχυνση της εφαρμογής.\newline \indent
Για την εκτέλεση πράξεων \ac{SIMD}, ο προγραμματιστής της εφαρμογής χρησιμοποιεί τα διαθέσιμα \textsl{intrinsics} που προσφέρονται από τους μεταγλωττιστές. Οι διαθέσιμες \ac{SIMD} συναρτήσεις χωρίζονται στις ακόλουθες κατηγορίες:
\begin{itemize}

\item{\textsl{Συναρτήσεις πρόσθεσης/αφαίρεσης}: Πρόσθεση, αφαίρεση, μερικά αθροίσματα, μέσος όρος.}

\item{\textsl{Συναρτήσεις πολλαπλασιασμού/διαίρεσης}: Πολλαπλασιασμός, διαίρεση, πράξεις υπολοίπου και πράξεις \textsl{modulo}.}

\item{\textsl{Συναρτήσεις αντιμετάθεσης και ολίσθησης}: Αναδιάταξη και μετακίνηση στοιχείων διανυσμάτων.}

\item{\textsl{Βασικές συναρτήσεις ενός τελεστέου}: Απόλυτη τιμή, στρογγυλοποίηση, αντιστροφή αριθμού.}

\item{\textsl{Λογικές Συναρτήσεις}.}

\item{\textsl{Συναρτήσεις σύγκρισης στοιχείων διανυσμάτων}.}

\item{\textsl{Μαθηματικές συναρτήσεις: Τριγωνομετρικές συναρτήσεις, συναρτήσεις λογαρίθμου και εκθετικές συναρτήσεις}.}

\end{itemize}
\indent
Τα σχήματα~\ref{figure:fig39} και~\ref{figure:fig310} παρουσιάζουν δύο ενδεικτικά παραδείγματα όπου γίνεται χρήση των συναρτήσεων \ac{SIMD}. Στο πρώτο σχήμα παρουσιάζεται μία διανυσματική πρόσθεση. Η συνάρτηση λαμβάνει ως ορίσματα δύο διανυσματικούς καταχωρητές και επιστρέφει έναν καταχωρητή με στοιχεία το άθροισμα των αντίστοιχων στοιχείων των τελεστέων. Στο δεύτερο σχήμα παρουσιάζεται μία πράξη διανυσματικής αναδιάταξης. Η αντίστοιχη συνάρτηση λαμβάνει ως ορίσματα τρεις καταχωρητές και επιστρέφει ως αποτέλεσμα έναν καταχωρητή ο οποίος έχει ως στοιχεία τα στοιχεία που έχουν επιλεγεί από τους δύο πρώτους καταχωρητές βάσει της μάσκας (\textsl{pattern}) που εμπεριέχεται στον τρίτο καταχωρητή.

\begin{figure}
\centering
\includegraphics[width=5in, height=2.5in]{Chapter3/figures/figure9.eps}
\caption{Παράδειγμα διανυσματικής πρόσθεσης.}
\label{figure:fig39}
\end{figure}

\begin{figure}
\centering
\includegraphics[width=5in, height=2.5in]{Chapter3/figures/figure10.eps}
\caption{Παράδειγμα διανυσματικής αναδιάταξης.}
\label{figure:fig310}
\end{figure}

\subsubsection{Μεταφορά Δεδομένων}
\label{subsubsection:subsub3322}
\indent
Το μέγεθος της τοπικής μνήμης των \acp{SPE} αλλά και η κατανεμημένη φύση του επεξεργαστή \textsl{Cell} δεν επιτρέπουν την αποθήκευση όλων των δεδομένων που απαιτούνται από τον κώδικα των \acp{SPE} στην τοπική μνήμη αυτών. Όπως αναφέρθηκε και στην ενότητα~\ref{subsection:sub322}, η μεταφορά δεδομένων τόσο από την κύρια μνήμη προς την τοπική μνήμη των \acp{SPE} και αντίστροφα όσο και μεταξύ των \acp{SPE} πραγματοποιείται μέσω του μηχανισμού \ac{DMA}. Οι εντολές που χρησιμοποιούνται είναι συγκεκριμένες εντολές του ελεγκτή μνήμης. Εντολές τύπου \textsl{get} χρησιμοποιούνται για την μεταφορά δεδομένων προς την τοπική μνήμη ενώ εντολές τύπου \textsl{put} χρησιμοποιούνται για μεταφορά δεδομένων από την τοπική μνήμη.\newline \indent
Αρκετές \textsl{data-intensive} εφαρμογές χρησιμοποιούν τα \acp{SPU} βάσει των παρακάτω βημάτων:
\begin{itemize}

\item{Κάθε \ac{SPU} λαμβάνει τα δεδομένα προς επεξεργασία από την κύρια μνήμη μέσω αιτήσεων \ac{DMA}.}

\item{Τα \acp{SPU} επεξεργάζονται τα δεδομένα.}

\item{Τα \acp{SPU} μεταφέρουν τα επεξεργασμένα πλέον δεδομένα από την \ac{LS} στην κύρια μνήμη του συστήματος.}

\end{itemize}
\indent Εάν αυτά τα βήματα εκτελούνται σειριακά, κάθε \ac{SPE} δαπανά ένα μεγάλο ποσοστό του χρόνου, αναμένοντας αδρανές για την ολοκλήρωση των μεταφορών. Όμως, η ασύγχρονη φύση των εντολών \ac{DMA} επιτρέπει την \textsl{παράλληλη} εκτέλεση αυτών των βημάτων. Η τεχνική αυτή ονομάζεται \textsl{multibuffering}. Η πιο συνήθης μορφή της τεχνικής του \textsl{multibuffering} είναι η τεχνική του \textsl{double-buffering}. Μία εφαρμογή που χρησιμοποιεί την τεχνική του \textsl{double-buffering} δεσμεύει δύο \textsl{buffers}, αντί ενός, για τα δεδομένα που πρόκειται να μεταφερθούν από και προς την \ac{LS}. Ενώ το \ac{SPU} επεξεργάζεται τα δεδομένα στον πρώτο \textsl{buffer}, τα δεδομένα στον δεύτερο \textsl{buffer} μεταφέρονται από ή προς την κύρια μνήμη. Έπειτα, οι ρόλοι των δύο \textsl{buffers} αντιστρέφονται: To \ac{SPU} επεξεργάζεται τα δεδομένα που έχουν μεταφερθεί στον δεύτερο \textsl{buffer} και παράλληλα δεδομένα μεταφέρονται από ή προς την \ac{LS} με χρήση του πρώτου \textsl{buffer}. Κατ' αυτό τον τρόπο είναι δυνατή η μεγιστοποίηση του χρόνου κατά τον οποίο εκτελούνται υπολογισμοί με την ταυτόχρονη ελαχιστοποίηση του χρόνου όπου ο επεξεργαστής αναμένει αδρανής για την ολοκλήρωση της μεταφοράς των δεδομένων. \newline \indent
Ο μηχανισμός \ac{DMA} είναι αρκετά απλός ώστε να διατηρηθεί η απλότητα στην αρχιτεκτονική του επεξεργαστή. Έτσι, μία εντολή \ac{DMA} πραγματοποιεί μεταφορά δεδομένων από συνεχόμενες περιοχές τόσο στην τοπική μνήμη όσο και στην κύρια μνήμη ενώ απουσιάζουν εξειδικευμένα χαρακτηριστικά όπως τα μεγέθη \textsl{stride} και \textsl{span} που ενσωματώνονται σε άλλους μηχανισμούς \ac{DMA}. Όμως, διατίθενται εντολές για συγχρονισμό μεταξύ πολλαπλών αιτήσεων \ac{DMA} που προέρχονται από το ίδιο \ac{SPE}.\newline \indent
Το μέγεθος των δεδομένων προς μεταφορά θα πρέπει να είναι ίσο με \(1, 2, 4, 8\) ή πολλαπλάσιο των \(16\ bytes\), με μέγιστο μέγεθος τα \(16\ KB\). Συν τοις άλλοις, οι δύο διευθύνσεις, μεταξύ των οποίων μεταφέρονται τα δεδομένα, θα πρέπει να εμφανίζουν την ίδια ευθυγράμμιση. Σε περίπτωση που το μέγεθος των δεδομένων είναι μικρότερο από \(16\ bytes\), οι δύο διευθύνσεις θα πρέπει να εμφανίζουν φυσική ευθυγράμμιση\footnote{\small Φυσική ευθυγράμμιση είναι η ευθυγράμμιση των δύο διευθύνσεων σε θέση που είναι πολλαπλάσια του μεγέθους των δεδομένων που μεταφέρονται.}, ενώ σε διαφορετική περίπτωση οι δύο διευθύνσεις θα πρέπει να έχουν ευθυγράμμιση σε διευθύνσεις που είναι πολλαπλάσιες του \(16\). Η βέλτιστη μεταφορά δεδομένων επιτυγχάνεται όταν και οι δύο διευθύνσεις βρίσκονται ευθυγραμμισμένες σε διεύθυνση η οποία είναι πολλαπλάσια των \(128\ bytes\). Οι απαιτήσεις ευθυγράμμισης είναι αρκετά περιοριστικές και εισάγουν επιπλέον καθυστερήσεις σε περίπτωση που το \textsl{memory access pattern} της εφαρμογής δεν τις ικανοποιεί.\newline \indent
Σε περίπτωση που η περιοχή της κύριας μνήμης δεν είναι συνεχόμενη, είναι δυνατή η χρήση του μηχανισμού των λιστών \ac{DMA}-\ac{DMA} \textsl{lists}. Οι λίστες \ac{DMA} ενσωματώνουν πολλαπλές μεμονωμένες αιτήσεις \ac{DMA} σε μία μόνο αίτηση, επιτρέποντας έτσι την αποφυγή του κόστους που συνεπάγεται η αρχικοποίηση του μηχανισμού \ac{DMA} για κάθε επιμέρους αίτηση. Όσον αφορά στο μέγιστο πλήθος αιτήσεων, αυτό ισούται με \(2048\) αιτήσεις, με μέγιστο μέγεθος δεδομένων προς μεταφορά ίσο με \(16\ KB\) για κάθε αίτηση. Όμως, και οι λίστες \ac{DMA} θα πρέπει να ικανοποιούν τους περιορισμούς ευθυγράμμισης του μηχανισμού \ac{DMA}.\newline \indent
Τέλος, εκτός από τον μηχανισμό των αιτήσεων \ac{DMA}, η επικοινωνία μεταξύ του \ac{PPE} και των \acp{SPE} επιτυγχάνεται μέσω του μηχανισμού των γραμματοκιβωτίων (\textsl{mailboxes}) και τον μηχανισμό των σημάτων (\textsl{signals}). Αυτοί οι μηχανισμοί ενδείκνυνται σε περίπτωση που το μέγεθος των μηνυμάτων προς ανταλλαγή δεν υπερβαίνει τα \(32\ bits\). Κάθε \ac{SPE} διαθέτει:
\begin{itemize}

\item{1 FIFO mailbox, τεσσάρων θέσεων, για τα εισερχόμενα μηνύματα.}

\item{1 mailbox, μίας θέσης, για εξερχόμενα μηνύματα.}

\item{1 mailbox, μίας θέσης, για εξερχόμενα μηνύματα προς το \ac{PPE} με διακοπή.}

\item{2 καταχωρητές μεγέθους \(4\ bytes\), με ειδοποίηση διακοπών, για την αποστολή μηνυμάτων στο αντίστοιχο \ac{SPE}.}

\end{itemize}

\subsubsection{Επικαλυπτόμενα Τμήματα Κώδικα}
\label{subsubsection:subsub3323}
\indent
Με δεδομένο το περιορισμένο μέγεθος της τοπικής μνήμης των \acp{SPE} και της αρχιτεκτονικής αυτής - ενοποιημένη μνήμη εντολών και δεδομένων - υπάρχει το ενδεχόμενο της ανεπάρκειας της μνήμης για την αποθήκευση ενός εκτελέσιμου για το \ac{SPE}. Σε αυτή την περίπτωση, τα επικαλυπτόμενα τμήματα κώδικα είναι ένας πολύ χρήσιμος μηχανισμός. Αυτή η προσέγγιση, τεμαχίζει τον κώδικα της εφαρμογής σε επιμέρους τμήματα ενώ καταλαμβάνεται ένα τμήμα της τοπικής μνήμης των \acp{SPE} για τον διαχειριστή των επικαλυπτόμενων τμημάτων. Ο προσδιορισμός των τμημάτων γίνεται από τον προγραμματιστή, ο οποίος παρέχει το κατάλληλο αρχείο σεναρίου (\textsl{script file}) στον \textsl{linker}. Κατά την εκτέλεση της εφαρμογής, ο διαχειριστής των τμημάτων είναι υπεύθυνος για την μεταφορά των τμημάτων από την κύρια μνήμη στην τοπική μνήμη τωv \acp{SPE}, όποτε αυτό είναι απαραίτητο. Η μεταφορά των τμημάτων απαιτεί έναν σημαντικό αριθμό από κύκλους μηχανής αλλά αυτός είναι ο μόνος και ο πιο αποδοτικός τρόπος για την εκτέλεση εφαρμογών για τα \acp{SPE} που απαιτούν περισσότερα από \(256\ KB\) μνήμης.

\subsection[3.3.3 Ο Προσομοιωτής]{Ο Προσομοιωτής}
\label{subsection:sub333}
\indent
Ο προσομοιωτής \textsl{IBM Full-System Simulator}, ο οποίος παρέχεται με το \textsl{Cell BE SDK}, είναι μία εφαρμογή λογισμικού που προσομοιώνει την συμπεριφορά ενός συστήματος με έναν επεξεργαστή \textsl{Cell BE}. Ο χρήστης μπορεί να εκκινήσει το λειτουργικό σύστημα \textsl{Linux} μέσω του \textsl{image} που παρέχεται και να εκτελέσει τις εφαρμογές στο σύστημα που προσομοιώνεται. Επίσης, ο προσομοιωτής υποστηρίζει την εκτέλεση \textsl{statically linked} προγραμμάτων χωρίς την εκτέλεση του λειτουργικού συστήματος, ενώ παρέχονται και διαγνωστικές λειτουργίες, όπως η λειτουργία αποσφαλμάτωσης. Το σχήμα~\ref{figure:fig311} παρουσιάζει το διάγραμμα της στοίβας προσομοίωσης για τον επεξεργαστή \textsl{Cell}.

\begin{figure}
\centering
\includegraphics[width=5in, height=2.5in]{Chapter3/figures/figure11.eps}
\caption{Η στοίβα προσομοίωσης για τον επεξεργαστή \textsl{Cell}.}
\label{figure:fig311}
\end{figure}

\indent
Η υποδομή του προσομοιωτή έχει σχεδιασθεί ώστε να είναι δυνατή η μοντελοποίηση του επεξεργαστή και της αρχιτεκτονικής σε διάφορα επίπεδα αφαίρεσης, τα οποία ποικίλουν από λειτουργική προσομοίωση του συστήματος μέχρι λεπτομερή προσομοίωση της συνολικής επίδοσης του συστήματος.
\begin{itemize}

\item{\textbf{Λειτουργική Προσομοίωση}: Αυτού του είδους η προσομοίωση μοντελοποιεί τα ορατά αποτελέσματα της εκτέλεσης των εντολών του προγράμματος, χωρίς να μοντελοποιεί τον χρόνο που απαιτείται για την εκτέλεση αυτών των εντολών. Για την εκτέλεση, γίνεται η υπόθεση ότι κάθε εντολή μπορεί να εκτελεστεί σε ένα σταθερό αριθμό κύκλων ρολογιού. Επίσης, οι προσπελάσεις της μνήμης είναι \textsl{σύγχρονες} και εκτελούνται σε σταθερό αριθμό κύκλων ρολογιού. Αυτό το μοντέλο προσομοίωσης είναι χρήσιμο για ανάπτυξη εφαρμογών και αποσφαλμάτωση αυτών όταν η ακριβής μέτρηση του χρόνου εκτέλεσης δεν είναι απαραίτητη.}

\item{\textbf{Λεπτομερής Προσομοίωση}: Για την ανάλυση της επίδοσης της εφαρμογής και του συστήματος χρησιμοποιείται η λεπτομερής προσομοίωση. Αυτού του είδους η προσομοίωση είναι αρκετά πιο χρονοβόρα από την λειτουργική προσομοίωση καθώς μοντελοποιούνται όλα τα στοιχεία του συστήματος, μέσω ενός κατάλληλου μοντέλου ανάλυσης. Η μοντελοποίηση των διαφορών καθυστερήσεων γίνεται δυναμικά ώστε να ληφθούν υπόψη τόσο ο χρόνος επεξεργασίας που απαιτείται όσο και οι διάφοροι περιορισμοί των πόρων του συστήματος.}

\end{itemize}
\indent
Η τρέχουσα έκδοση του προσομοιωτή υποστηρίζει την προσομοίωση σε επίπεδο κύκλου ρολογιού (\textsl{cycle-accurate simulation}) για όλο το σύστημα εκτός από το \ac{PPE}. Τα μοντέλα που προσομοιώνουν την λειτουργία των \acp{SPE} μοντελοποιούν με μεγάλη ακρίβεια την ροή των εντολών καθώς αυτές διέρχονται από τα δύο \textsl{pipelines} των \acp{SPE}. Αυτό το χαρακτηριστικό επιτρέπει την λήψη ενός μεγάλου αριθμού από χρήσιμες πληροφορίες που αφορούν στην εκτέλεση του προγράμματος, όπως για παράδειγμα τον αριθμό των εντολών που εκτελέστηκαν παράλληλα, τον αριθμό των \textsl{pipeline stalls} και την αιτία αυτών, τον αριθμό των λανθασμένων προβλέψεων διακλάδωσης, κ.α.

\subsection[3.3.4 Accelerated Library Framework (ALF)]{Accelerated Library Framework (ALF)}
\label{subsection:sub334}
\indent
Το \ac{ALF} είναι μία προγραμματιστική διεπαφή (\ac{API}) που δημιουργήθηκε για να επιταχύνει την διαδικασία ανάπτυξης εφαρμογών, παρέχοντας ένα επίπεδο αφαίρεσης των παράλληλων εφαρμογών σε συστήματα πολλαπλών πυρήνων. Η υλοποίηση του \textsl{framework} επικεντρώνει στην επίλυση προβλημάτων με \textsl{παραλληλισμό μεταξύ δεδομένων} (\textsl{data parallelism}) σε υβριδικά συστήματα \textsl{host-accelerator}. Το \ac{ALF} υποστηρίζει το προγραμματιστικό μοντέλο \ac{MIMD}, όπου είναι δυνατή η ταυτόχρονη εκτέλεση πολλαπλών προγραμμάτων σε πολλαπλούς επεξεργαστές. Τα σημαντικότερα χαρακτηριστικά αυτού του \textsl{framework} είναι η διαχείριση της μεταφοράς των δεδομένων, η διαχείριση των παράλληλων εργασιών, η διαχείριση των πολλαπλών \textsl{buffers} για την τεχνική του \textsl{multi-buffering} όπως επίσης και ο τεμαχισμός των δεδομένων για κάθε επεξεργαστή.

\begin{figure}
\centering
\includegraphics[width=5in, height=3in]{Chapter3/figures/figure12.eps}
\caption{Επισκόπηση των συστατικών στοιχείων του \textsl{ALF}.}
\label{figure:fig312}
\end{figure}

\indent
Το σχήμα~\ref{figure:fig312} παρουσιάζει μία επισκόπηση των συστατικών στοιχείων του \ac{ALF}. Στον κεντρικό επεξεργαστή (\textsl{host processor}) γίνεται ο τεμαχισμός των δεδομένων εισόδου και των αντίστοιχων δεδομένων εξόδου σε μικρότερες μονάδες που αποκαλούνται \textsl{work blocks}. Με το παρεχόμενο \ac{API}, τα \textsl{work blocks} αποθηκεύονται στην αντίστοιχη \textsl{work queue} και εν συνεχεία κατανέμονται στους επιμέρους επεξεργαστές. Έπειτα, οι επεξεργαστές εκτελούν την επεξεργασία στα \textsl{work blocks} και επιστρέφουν τα δεδομένα εξόδου στον κεντρικό επεξεργαστή. Το \ac{ALF} υποστηρίζει μόνο σύνολα δεδομένων εισόδου τα οποία μπορούν να τεμαχιστούν σε \textsl{work blocks} με κατάλληλο μέγεθος ώστε να είναι δυνατή η αποθήκευση αυτών στην τοπική μνήμη των \acp{SPE}.\newline \indent
Συνοψίζοντας, το \ac{ALF} διευκολύνει τους προγραμματιστές των εφαρμογών καθώς αναλαμβάνει την διαχείριση της μεταφοράς των δεδομένων, την διαχείριση των πόρων και των εργασιών αλλά και τα διάφορα θέματα συγχρονισμού που ενδέχεται να υπάρχουν. Παρόλα αυτά, η υλοποίηση ενός καλού τεμαχισμού των δεδομένων και η βελτιστοποίηση των υπολογισμών είναι καθήκον του προγραμματιστή. Ο τεμαχισμός των δεδομένων είναι πολύ σημαντικός τόσο για το \ac{ALF} όσο και για την αρχιτεκτονική του επεξεργαστή, καθώς θα πρέπει να ληφθεί υπόψη το περιορισμένο μέγεθος της τοπικής μνήμης των \acp{SPE} και η μνήμη που απαιτείται για το \textsl{double-buffering}. Τέλος, οι υπολογιστικοί πυρήνες (\textsl{computational kernels}) θα πρέπει να βελτιστοποιηθούν από τον προγραμματιστή για την αρχιτεκτονική των \acp{SPE}.








