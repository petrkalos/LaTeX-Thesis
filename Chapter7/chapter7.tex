\chapter{Επίλογος}
\label{chapter:chap7}
\indent
Η πρωτοφανής υπολογιστική δύναμη του ετερογενούς επεξεργαστού πολλαπλών πυρήνων \acf{CBEA} επέτρεψε την ανάπτυξη λύσεων σε λογισμικό για την επίλυση υπολογιστικά απαιτητικών αλγορίθμων με απαιτήσεις πραγματικού χρόνου (\textsl{real-time constraints}), οι οποίοι μέχρι πρότινος απαιτούσαν ειδική υποστήριξη από το υλικό. Η παρούσα εργασία επικεντρώνει στην ανάπτυξη ενός αλγορίθμου για την διόρθωση της παραμόρφωσης που προκαλείται από ευρυγώνιους φακούς σε πραγματικό χρόνο. Η διόρθωση παραμόρφωσης που προκαλείται από τους ευρυγώνιους φακούς είναι μία εφαρμογή \textsl{στρέβλωσης εικόνας} (\textsl{image warping application}) με εφαρμογή σε διάφορους επιστημονικούς τομείς. Κύριο χαρακτηριστικό αυτής της εφαρμογής είναι το στατικό αλλά μη κανονικό πρότυπο πρόσβασης στην μνήμη το οποίο καθιστά την προφόρτωση των δεδομένων αρκετά δύσκολη.\newline \indent
Σε αυτή την εργασία περιγράψαμε και αξιολογήσαμε πειραματικά τα αλγοριθμικά βήματα και τις οδηγούμενες από την αρχιτεκτονική του επεξεργαστή βελτιστοποιήσεις που απαιτούνται για την επίτευξη διόρθωσης σε πραγματικό χρόνο (\(30\ fps\)) για μία ακολουθία εικόνων που βασίζεται στο \textsl{format RGB}, με μέγεθος \textsl{frame} εισόδου \(2592x1944\) και μέγεθος \textsl{frame} εξόδου \(640x480\). Επίσης, διερευνήσαμε τα αποτελέσματα στην απόδοση της εφαρμογής διαφόρων \textsl{coarse} και \textsl{fine-grain} βελτιστοποιήσεων. Ξεκινώντας από απόδοση \(4.17\ fps\) σε έναν επεξεργαστή \textsl{Core2 Duo}, εκτελέσαμε και βελτιστοποιήσαμε σταδιακά τον κώδικα της εφαρμογής στον επεξεργαστή \ac{CBE} με αποτέλεσμα την επίτευξη απόδοσης ίση με \(30\ fps\). Κατά την διάρκεια των βελτιστοποιήσεων, οι πιο προσοδοφόρες βελτιστοποιήσεις ήταν η απεικόνιση των υπολογιστικά πολυπλοκότερων πυρήνων της εφαρμογής στα \acp{SPE} (\textsl{task offloading}), η διανυσματοποίηση των υπολογισμών (\textsl{vectorization}) και η εξάλειψη των εντολών διακλάδωσης (\textsl{branch elimination}).\newline \indent
Συν τοις άλλοις, διερευνήσαμε αρκετές σχεδιαστικές επιλογές για την εκτίμηση της επίδρασης εναλλακτικών μετασχηματισμών του λογισμικού στην απόδοση της εφαρμογής. Όπως προέκυψε, η βέλτιστη απόδοση επιτυγχάνεται όταν η διαδικασία της \textsl{αντίστροφης απεικόνισης} (\textsl{Inverse Mapping}) υλοποιούνταν στο \ac{PPE} (με χρήση των επεκτάσεων \textsl{AltiVec}) και έπειτα πραγματοποιούνταν η διανομή των κλασματικών συντεταγμένων στα \acp{SPE}. Αυτή η προσέγγιση επιτυγχάνει τον ελάχιστο χρόνο εκτέλεσης, ο οποίος ισούται με \(0.033\ secs/frame\). Επίσης, πραγματοποιήσαμε μία λεπτομερή ανάλυση της απόδοσης της εφαρμογής στον επεξεργαστή \textsl{Cell}, σε έναν επεξεργαστή \textsl{Core2 Quad} από την \textsl{Intel} και σε μία \textsl{Virtex-4 LX80} \ac{FPGA}.\newline \indent
Μία ενδιαφέρουσα διαπίστωση που προέκυψε από την υλοποίηση της εφαρμογής στον επεξεργαστή \ac{CBEA} είναι ο σημαντικός βαθμός μη αυτόματης δρομολόγησης των εντολών που απαιτείται από την προγραμματιστή ώστε να μειωθούν τα παγώματα του \textsl{pipeline} που οφείλονται τόσο σε εξαρτήσεις μεταξύ εντολών όσο και στην μη βέλτιστη δρομολόγηση των εντολών για την εκμετάλλευση του \textsl{dual-issue pipeline}. Αυτό επιδεικνύει την ανάγκη για πιο ώριμη τεχνολογία μεταγλωττιστών, οι οποίοι θα μπορούν να παράγουν βελτιστοποιημένο κώδικα.\newline \indent 
Ένα ακόμη σημαντικό συμπέρασμα είναι οι ελάχιστες δυνατότητες που προσφέρει ο μηχανισμός \ac{DMA} της αρχιτεκτονικής \ac{CBEA}. Κυρίως για εφαρμογές που ανήκουν στο \textsl{streaming domain} και χαρακτηρίζονται από ένα μη κανονικό πρότυπο προσπέλασης στην μνήμη και τον μεγάλο όγκο δεδομένων που μεταφέρονται, η έλλειψη χαρακτηριστικών όπως τα μεγέθη \textsl{skip} ή \textsl{stride} καθιστούν την μεταφορά των δεδομένων ένα προκλητικό εγχείρημα για τον προγραμματιστή της εφαρμογής.\newline \indent 
Τέλος, όπως προκύπτει από την πλειονότητα των βελτιστοποιήσεων, ο προγραμματιστής των εφαρμογών για επεξεργαστές πολλαπλών πυρήνων, και κυρίως για ετερογενείς επεξεργαστές πολλαπλών πυρήνων, θα πρέπει να έχει μία άριστη γνώση της αρχιτεκτονικής του συστήματος ώστε να μπορεί να εφαρμόζει τους απαραίτητους μετασχηματισμούς για την βελτιστοποίηση του κώδικα της εκάστοτε εφαρμογής. 