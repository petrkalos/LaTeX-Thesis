\chapter{Σχεδιαστικά Trade-offs και Ανάλυση Ευαισθησίας}
\label{chapter:chap5}

Η ανάπτυξη μίας εφαρμογής με εγγενή πολυεπίπεδο παραλληλισμό σε μία πολύπλοκη και ετερογενή παράλληλη αρχιτεκτονική, όπως ο επεξεργαστή \ac{CBE}, απαιτεί την λήψη διαφόρων σχεδιαστικών αποφάσεων από τον προγραμματιστή. Σε αυτό το κεφάλαιο αναλύονται οι διάφοροι \textsl{αντιπραγματισμοί} (\textsl{trade-offs}) που αντιμετωπίσθηκαν κατά την ανάπτυξη της εφαρμογής και πραγματοποιείται μία ανάλυση της ευαισθησίας της εφαρμογής σε διάφορες σχεδιαστικές παραμέτρους.

\section{Μέγεθος των Tiles}
\label{section:sect51}
\indent
Το μέγεθος και το σχήμα των \textsl{tiles} είναι ένας πολύ σημαντικός παράγοντας σε εφαρμογές όπου χρησιμοποιείται η τεχνική του \textsl{tiling}. Το μέγεθος του \textsl{tile} θα πρέπει να είναι αρκετά μεγάλο ώστε να εκμεταλλευτεί όλη την διαθέσιμη μνήμη \textsl{cache}, ή την διαθέσιμη \ac{LS} στην αρχιτεκτονική του \ac{CBEA}, προκειμένου να μεγιστοποιηθεί η επαναχρησιμοποίηση των δεδομένων και το σύνολο εργασίας, με την ταυτόχρονη ελαχιστοποίηση του κόστους που απαιτείται για την επικοινωνία. Η τελευταία παρατήρηση βασίζεται στο γεγονός ότι τεμαχισμός της εικόνας σε μικρότερο αριθμό από \textsl{tiles} έχει ως επιπλέον αποτέλεσμα την ελαχιστοποίηση του κόστους που απαιτείται για την αρχικοποίηση των απαιτούμενων αιτήσεων \ac{DMA} για την μεταφορά των \textsl{tiles}. Συν τοις άλλοις, το μέγεθος των \textsl{tiles} θα πρέπει να είναι μικρό λόγω της περιορισμένης χωρητικότητας της \textsl{LS} των \acp{SPE} αλλά και της κύρτωσης των \textsl{tiles} της εικόνας εισόδου.\newline \indent
Όμως, στην περίπτωση της διόρθωσης της παραμόρφωσης που προκαλείται από την χρήση ευρυγώνιων φακών, εκτός από το μέγεθος της τοπικής μνήμης των \acp{SPE} και του \textsl{αποτυπώματος της μνήμης} (\textsl{memory footprint}) των άλλων δεδομένων που πρέπει να αποθηκευθούν, θα πρέπει να ληφθούν υπόψη και άλλες παράμετροι του προβλήματος. Στις περισσότερες περιπτώσεις, το τμήμα της εικόνας εισόδου που πρέπει να μεταφερθεί είναι κυρτό και όχι ορθογώνιο, όπως παρουσιάζεται στο Σχήμα~\ref{figure:fig41}. Δεδομένων των περιορισμών ευθυγράμμισης στην πρόσβαση της μνήμης και των περιορισμένων δυνατοτήτων του μηχανισμού \ac{DMA}, ο χειρισμός των κυρτών περιοχών καθίσταται αρκετά περίπλοκος και εισάγει επιπλέον καθυστερήσεις στην εκτέλεση της εφαρμογής. Για την αποφυγή αυτού του κόστους, κάθε \ac{SPE} μεταφέρει την ορθογώνια περιοχή που περικλείει το αντίστοιχο \textsl{tile}.\newline \indent 
Συν τοις άλλοις, επιπλέον περιορισμοί επιβάλλονται από τα χαρακτηριστικά της αρχιτεκτονικής του επεξεργαστή στο σχήμα και στο μέγεθος της ορθογώνιας περιοχής που περικλείει κάθε \textsl{tile}. Επί παραδείγματι, το μήκος της ορθογώνιας περιοχής θα πρέπει να είναι πολλαπλάσιο του \(4\), ώστε να διευκολυνθεί η διαδικασία της διανυσματοποίησης των υπολογισμών.\newline \indent 
Επίσης, κάθε γραμμή από \textsl{pixels} της περιοχής θα πρέπει να ξεκινά από διεύθυνση η οποία είναι πολλαπλάσια των \(16B\), καθώς αυτός είναι ο ελάχιστος περιορισμός ευθυγράμμισης που θα πρέπει να ικανοποιείται ώστε να είναι δυνατή η μεταφορά δεδομένων με μέγεθος μεγαλύτερο των \(16B\) μέσω του μηχανισμού \ac{DMA}.\newline \indent 
Λόγω της κυρτότητας που παρουσιάζει η εικόνα εισόδου, \textsl{tiles} πολύ μικρού μεγέθους έχουν ως αποτέλεσμα την μεταφορά ενός μεγάλου αριθμού από \textsl{pixels} τα οποία βρίσκονται εκτός της περιοχής ενδιαφέροντος, αλλά πρέπει να μεταφερθούν καθώς βρίσκονται εντός της ορθογώνιας περιοχής. Τα επιπλέον \textsl{pixels} θα πρέπει να διορθωθούν οπότε αυξάνεται τόσο ο χρόνος μεταφοράς όσο και ο χρόνος της επεξεργασίας των \textsl{tiles}.\newline \indent 
Το ίδιο αποτέλεσμα έχουν και \textsl{tiles} τα οποία είναι πολύ εκτεταμένα προς την μία ή την άλλη κατεύθυνση. Γι' αυτό τον λόγο δεν ήταν δυνατή και η ικανοποίηση του περιορισμού ευθυγράμμισης των \(128\ bytes\) για την βέλτιστη μεταφορά δεδομένων, όπως αναφέρεται στην υποενότητα~\ref{subsubsection:subsub3322}. Σε αυτή την περίπτωση, θα έπρεπε το μήκος της κάθε ορθογώνιας περιοχής να είναι πολλαπλάσιο του \(128\), κάτι που οδηγεί στα προαναφερθέντα, μη επιθυμητά αποτελέσματα.\newline
\indent
Το διάγραμμα (a) της Εικόνας~\ref{figure:fig51} παρουσιάζει το επιπλέον κόστος (ο αριθμός των \textsl{pixels} που μεταφέρονται προς τον συνολικό αριθμό των \textsl{pixels} του \textsl{frame} - \textsl{transfer overhead}) για την μεταφορά των \textsl{pixels} εισόδου που απαιτούνται για την διόρθωση ενός \textsl{frame}, για όλα τα δυνατά μεγέθη και σχήματα των \textsl{tiles} της εικόνας εξόδου, βάσει των προαναφερθέντων περιορισμών που επιβάλλονται. Το διάγραμμα (b) απεικονίζει τον κανονικοποιημένο\footnote{\small Ο χρόνος εκτέλεσης είναι κανονικοποιημένος αναφορικά με τον βέλτιστο χρόνο εκτέλεσης που επιτυγχάνεται όταν το \textsl{frame} εξόδου αποτελείται από \(5\) \textsl{tiles} στην οριζόντια κατεύθυνση (\textsl{x-dimension}) και από \(20\) \textsl{tiles} στην κάθετη διεύθυνση.} χρόνο εκτέλεσης της εφαρμογής με χρήση των \(6\) διαθέσιμων \acp{SPE} του \textsl{PS3}. Οι μη εφικτοί συνδυασμοί λόγω των διαφόρων περιορισμών της αρχιτεκτονικής του επεξεργαστή \ac{CBEA} επισημειώνονται με X.\newline \indent 
Τα δύο διαγράμματα καταδεικνύουν μία στενή σχέση μεταξύ του χρόνου εκτέλεσης της εφαρμογής και του μεγέθους των δεδομένων που μεταφέρονται για τα διάφορα μεγέθη των \textsl{tiles}. Επιπλέον, είναι εμφανές ότι \textsl{tiles} με πολύ μικρό μέγεθος ή \textsl{tiles} τα οποία είναι πολύ εκτεταμένα προς την μία ή την άλλη κατεύθυνση (συνδυασμοί που αντιστοιχούν στην άνω αριστερή ή στην κάτω δεξιά γωνία) δεν είναι επιθυμητά. Το ποσοστό των επιπλέον \textsl{pixels} που μεταφέρονται κυμαίνεται από \(9\%\) έως \(331\%\). Η δυνητική αύξηση του χρόνου εκτέλεσης της εφαρμογής, αναφορικά με τον χρόνο εκτέλεσης της εφαρμογής για το βέλτιστο μέγεθος των \textsl{tiles} ισούται με \(22\%\). Βάσει αυτών των παρατηρήσεων και των αποτελεσμάτων, ο τεμαχισμός του \textsl{frame} εξόδου που χρησιμοποιείται αποτελείται από \(5\) \textsl{tiles} στην οριζόντια κατεύθυνση (\textsl{x-dimension}) και \(20\) \textsl{tiles} στην κάθετη διεύθυνση (\textsl{y-dimension}).
\begin{figure}
%\centering
\hspace{-0.5in}\subfloat[]{\includegraphics[width=3.0in, height=2.5in]{Chapter5/figures/figure1a.eps}}
\hspace{-0.1in}\subfloat[]{\includegraphics[width=3.0in, height=2.5in]{Chapter5/figures/figure1b.eps}}
\caption{Αριθμός \textsl{pixels} που μεταφέρονται προς τον συνολικό αριθμό των \textsl{pixels} του \textsl{frame} (διάγραμμα (a)) και κανονικοποιημένος χρόνος εκτέλεσης της εφαρμογής (διάγραμμα (b)).}
\label{figure:fig51}
\end{figure}
\section{Επιμερισμός του Κόστους της Αντίστροφης Απεικόνισης}
\label{section:sect52}
\indent
Η \textsl{αντίστροφη απεικόνιση}, δηλαδή η διαδικασία του υπολογισμού για κάθε \textsl{pixel} του \textsl{frame} εξόδου του αντιστοίχου κλασματικού \textsl{pixel} στο \textsl{frame} εισόδου, είναι μία διαδικασία που συγκαταλέγεται στους υπολογιστικά πολυπλοκότερους πυρήνες της εφαρμογής (Σχήμα~\ref{figure:fig42}). Όπως προαναφέρθηκε, η αντιστοιχία των συντεταγμένων των \textsl{pixels} μεταξύ εισόδου και εξόδου εξαρτάται αποκλειστικά από την \textsl{περιοχή ενδιαφέροντος} (\ac{ROI}) και το \textsl{πεδίο θέασης} (\ac{FoV}). Αυτές οι δύο παράμετροι του αλγορίθμου διόρθωσης μπορούν να αλλαχθούν διαδραστικά κατά την εκτέλεση του αλγορίθμου. Αυτό όμως είναι κάτι που είτε δεν συμβαίνει ποτέ είτε συμβαίνει με πολύ μικρή συχνότητα σε τυπικά σενάρια χρήσης των εφαρμογών \textsl{videoconferencing}. Επομένως, μπορούμε να επιμερίσουμε το κόστος για την διαδικασία της \textsl{αντίστροφης απεικόνισης} μέσω της εκτέλεσης αυτού του υπολογισμού μόνο για \(1\) \textsl{frame} και της επαναχρησιμοποίησης των συντεταγμένων που παράγονται στα επόμενα \textsl{frames}, έως ότου μεταβληθεί κάποια από τις προαναφερθείσες παραμέτρους.\newline \indent 
Αυτή η επαναχρησιμοποίηση έχει ως αποτέλεσμα την αύξηση των απαιτήσεων σε μνήμη της εφαρμογής: το μέγεθος της δομής δεδομένων που χρησιμοποιείται για την αποθήκευση των συντεταγμένων ισούται με \(4.8\ MB\) καθώς πρέπει να αποθηκευθούν \(1280x960\) ζεύγη αριθμών κινητής υποδιαστολής απλής ακρίβειας. Μία τέτοιου μεγέθους δομή δεδομένων δεν θα μπορούσε να αποθηκευθεί στην τοπική μνήμη των \acp{SPE}. Βάσει αυτών, αξιολογήθηκε την εναλλακτική λύση της αποθήκευσης των συντεταγμένων στην κύρια μνήμη και την μεταφορά των αντιστοίχων \textsl{blocks} κλασματικών συντεταγμένων στην τοπική μνήμη του κάθε \ac{SPE}, βάσει του \textsl{tile} της εικόνας εξόδου που αυτό επεξεργάζεται.\newline \indent
Αυτό το \textsl{trade-off} που εμφανίζεται μεταξύ υπολογισμών και επικοινωνίας παρουσιάζει ιδιαίτερο ενδιαφέρον, δεδομένης της μεγάλης υπολογιστικής δύναμης του \ac{CBEA}. Χρησιμοποιώντας την βιβλιοθήκη \textsl{SPU Timer Library} \cite{CellTimersLib} βρέθηκε ότι οι αυξημένες απαιτήσεις σε επικοινωνία, ακόμη και μετά την εφαρμογή της τεχνικής του \textsl{double-buffering}, είχαν ως αποτέλεσμα τον τετραπλασιασμό του αριθμού των κύκλων ρολογιού που η εφαρμογή δαπανά κατά την αναμονή για την ολοκλήρωση των αιτήσεων \ac{DMA}. Παρά την αύξηση του χρόνου που απαιτείται για την ολοκλήρωση των αιτήσεων, ο χρόνος εκτέλεσης της εφαρμογής για μία ακολουθία \(10\) \textsl{frames} μειώθηκε κατά μέσο όρο σε \(0.045\ sec/frame\) όταν χρησιμοποιούνταν και τα \(6\) διαθέσιμα \acp{SPE} του \textsl{PS3}, αυξάνοντας έτσι την απόδοση της εφαρμογής σε \(22\) \(frames/sec\).\newline \indent 
Η τροποποιημένη έκδοση της εφαρμογής εκτελέστηκε στον προσομοιωτή χρησιμοποιώντας και τα \(8\) διαθέσιμα \acp{SPE} αυτού. Με χρήση των δυνατοτήτων \textsl{profiling} υπολογίσθηκε ότι απαιτούνταν \(0.033\ secs\) για την διόρθωση ενός \textsl{frame}, το οποίο μεταφράζεται σε \(30\) \(frame/sec\). Αυτός ο χρόνος εκτέλεσης αντιστοιχεί σε μία επιτάχυνση ίση με \(7.27x\) αναφορικά με την πολυνηματική έκδοση της εφαρμογής στον επεξεργαστή \textsl{Core2 Duo}. Αυτή είναι και η τροποποίηση που επιτρέπει την εκτέλεση της εφαρμογής σε \textsl{πραγματικό χρόνο}. Το Σχήμα~\ref{figure:fig52} παρουσιάζει τον αντίστοιχο ψευδοκώδικα για την έκδοση της εφαρμογής μετά από αυτή την τροποποίηση.\newline \indent
Μία άλλη παράμετρος του \textsl{trade-off} μεταξύ υπολογισμών και επικοινωνίας είναι ο υπολογισμός εκ των προτέρων και η αποθήκευση στην κύρια μνήμη των συντελεστών \(U_{i}(s), V_{j}(t)\) της παρεμβολής αντί των συντεταγμένων \(s\) και \(t\). Αυτοί οι συντελεστές εξαρτώνται μόνο από τις τιμές των συντεταγμένων \(s\) και \(t\) οπότε είναι δυνατός ο υπολογισμός αυτών για μία φορά στο \ac{PPE} και η μεταφορά αυτών στα \acp{SPE} για την εκτέλεση της \textsl{bicubic} παρεμβολής. Το μειονέκτημα αυτής της προσέγγισης είναι ότι πλέον απαιτείται η μεταφορά και η αποθήκευση στα \acp{SPE} οχτώ συντελεστών, δηλαδή οχτώ αριθμών κινητής υποδιαστολής απλής ακρίβειας. Αυτό θα απαιτούσε τον τεμαχισμό της εικόνας σε \textsl{tiles} μικρότερου μεγέθους για την αύξηση του διαθέσιμου χώρου στην \ac{LS} των \acp{SPE}, με τα μη επιθυμητά αποτελέσματα που αυτό έχει στον χρόνο εκτέλεσης της εφαρμογής, όπως αναλύθηκε στην ενότητα~\ref{section:sect51}.
\begin{figure}
\centering
\begin{small}
\begin{algorithmic}[1]
\STATE \COMMENT{\textbf{Input}: The frames (in the wide-angle space) to be corrected}
\STATE \COMMENT{\textbf{Output}: The corrected frames (in the perspective space)}
\STATE Partition the output frame to blocks
\FORALL{frames}
	\IF{FoV has changed}
		\FORALL{pixels in the output frame}
			\STATE Compute the corresponding fractional position in the input frame ({\textbf{ InverseMapping()}})
		\ENDFOR
		\STATE Calculate the area of the input frame required for the calculation of each output frame block
	\ENDIF
	\STATE \COMMENT {Output blocks are statically partitioned to SPEs and processed concurrently on different SPEs}
	\FORALL{output blocks $i$}
		\STATE Fetch (async. DMA) input data for output block $i+1$
		\STATE Fetch (async. DMA) fractional coordinates in the wide-angle space for each pixel of the output block $i+1$
		\STATE Store (async. DMA) output block $i-1$
		\FORALL{pixels in output block $i$}
			\STATE Interpolate the pixel value at that fractional position ({\textbf{ BicubicInterpolation(), SIMDized}})
		\ENDFOR
		\STATE Apply a $2$-D low-pass filter to resize the output block ({\textbf{ LPF(), SIMDized}})
	\ENDFOR
\ENDFOR
\end{algorithmic}
\end{small}
\caption{\label{figure:fig52} Ψευδοκώδικας για τον αλγόριθμο διόρθωσης (τελική, βελτιστοποιημένη έκδοση).}
\end{figure}
\section{Προ-επεξεργασία των Δεδομένων Εισόδου στο PPE}
\label{section:sect53}
\indent
Όπως προαναφέρθηκε, οι μη ευθυγραμμισμένες προσβάσεις στην μνήμη είναι ένας παράγοντας που επηρεάζει σημαντικά την απόδοση στην εκτέλεση του κώδικα των \acp{SPE}. Αυτού του είδους οι προσβάσεις είναι χαρακτηριστικό των εφαρμογών που εμπεριέχουν υπολογισμούς με χρήση \textsl{stencil} (στην παρούσα εφαρμογή ως \textsl{stencil} νοείται η \(4x4\) γειτονιά), όπου το \textsl{stencil} διατρέχει τα δεδομένα εισόδου. Τα δεδομένα κάθε \textsl{frame} εισόδου οργανώνονται ως ένας \(2D\) πίνακας από \textsl{pixels} με τρία \textsl{bytes} για κάθε \textsl{pixel}. Kάθε \textsl{byte} αποτελεί την τιμή για την αντίστοιχη συνιστώσα χρώματος στο \textsl{RGB format}. Ο κώδικας που χρησιμοποιείται για την φόρτωση των δεδομένων στους καταχωρητές των \acp{SPE} εκτελεί την ακόλουθη διαδικασία:

\begin{itemize}

\item{Αποπολύπλεξη των \textsl{bytes} που αντιστοιχούν σε διαφορετικές συνιστώσες χρώματος, καθώς κάθε συνιστώσα χρώματος φορτώνεται σε έναν διαφορετικό καταχωρητή των \(128\ bits\) ώστε να είναι δυνατή η επεξεργασία αυτής ανεξάρτητα από τις άλλες συνιστώσες.}

\item{Μετατροπή (αναβάθμιση) των \(8-bit\) ακεραίων σε \(32-bits\) αριθμούς κινητής υποδιαστολής για την εκτέλεση των αντίστοιχων πράξεων.}

\item{Αποθήκευση σε έναν καταχωρητή τεσσάρων τιμών που αντιστοιχούν σε μία συνιστώσα χρώματος και σε τέσσερα διαδοχικά \textsl{pixels}.}

\end{itemize}
\indent
Για την ελαχιστοποίηση της προ-επεξεργασίας των δεδομένων στα \acp{SPE}, διερευνήσαμε την προ-επεξεργασία των δεδομένων στο \ac{PPE}. Τα δεδομένα εισόδου μετατρέπονται σε αριθμούς κινητής υποδιαστολής και αποθηκεύονται στην κύρια μνήμη κατά την φόρτωση του \textsl{frame} ενώ πραγματοποιείται και η απαραίτητη αποπολύπλεξη των διαφορετικών συνιστωσών χρώματος, όπου κάθε συνιστώσα αποθηκεύεται σε έναν ξεχωριστό \(1D\) πίνακα. Με αυτή την προεπεξεργασία η πιθανότητα ευθυγραμμισμένης φόρτωσης δεδομένων σε έναν καταχωρητή είναι ίση με \(25\%\). Για τον διπλασιασμό της πιθανότητας ευθυγραμμισμένης πρόσβασης, δημιουργήσαμε ένα δεύτερο αντίγραφο των \textsl{tiles} εισόδου στο οποίο αποθηκεύουμε τα ολισθημένα κατά δύο θέσεις αριστερά δεδομένα του πρώτου μισού του πίνακα. Με αυτό το σχήμα πλεονασμού επιτυγχάνεται πιθανότητα ευθυγραμμισμένης πρόσβασης ίση με \(50\%\).\newline \indent 
Τα κύρια μειονεκτήματα από αυτή την προσέγγιση είναι ο πολλαπλασιασμός με έναν παράγοντα ίσο με \(8\) τόσο του απαιτούμενου αποθηκευτικού χώρου (\textsl{memory footprint}) του κάθε \textsl{tile} στην \ac{LS} των \acp{SPE} όσο και του όγκου των δεδομένων για τα \textsl{tiles} εισόδου που πρέπει να μεταφερθούν προς επεξεργασία στα \acp{SPE}.\newline \indent
Η ανωτέρω τροποποίηση είχε ως αποτέλεσμα την αύξηση του χρόνου εκτέλεσης της εφαρμογής κατά \(30\%\). Για την εύρεση του κύριου περιοριστικού παράγοντα στην απόδοση της εφαρμογής χρησιμοποιήθηκε ο προσομοιωτής συστήματος. Όπως προέκυψε, η αύξηση του χρόνου εκτέλεσης οφειλόταν στην μη επικάλυψη του αυξημένου χρόνου που απαιτούνταν για την μεταφορά των δεδομένων με την εκτέλεση υπολογισμών. Συν τοις άλλοις, το επιπλέον κόστος του κώδικα για την προ-επεξεργασία των δεδομένων, καθιστούσε το \ac{PPE} τον κύριο ανασχετικό παράγοντα της εφαρμογής, με αποτέλεσμα τον περαιτέρω περιορισμό της απόδοσης της εφαρμογής.

\section{Χρήση DMA Lists για την Μεταφορά των Δεδομένων}
\label{section:sect54}
\indent
Ο μηχανισμός που χρησιμοποιείται για την μεταφορά των δεδομένων του κάθε \textsl{tile} από το \ac{PPE} στο αντίστοιχο \ac{SPE} είναι αυτός των ασύγχρονων αιτήσεων \ac{DMA}. Για κάθε γραμμή του εκάστοτε \textsl{tile} πραγματοποιείται μία ξεχωριστή αίτηση ενώ αιτήσεις για γραμμές που ανήκουν στο ίδιο \textsl{tile} λαμβάνουν το ίδιο \textsl{αναγνωριστικό αίτησης} (\textsl{tag id}). Καθώς ο μηχανισμός των αιτήσεων \ac{DMA} είναι ασύγχρονος, αυτό το αναγνωριστικό χρησιμοποιείται για την αναμονή προς ολοκλήρωση των αντιστοίχων αιτήσεων. Αυτές οι συνεχόμενες αιτήσεις εισάγουν ένα επιπλέον κόστος στην εκτέλεση της εφαρμογής, το οποίο προέρχεται από τον χρόνο που απαιτείται για την αρχικοποίηση του μηχανισμού. Επίσης, σε πολλές περιπτώσεις, ο περιορισμένος αποθηκευτικός χώρος που διαθέτει το κάθε \ac{SPE} για την αποθήκευση των εκκρεμών αιτήσεων ενδέχεται να μην επαρκεί.\newline \indent 
Για την αποφυγή αυτού του κόστους, εκμεταλλευτήκαμε την συνέχεια του χώρου μνήμης στην \ac{LS} όπου αποθηκεύονται τα \textsl{pixels} για κάθε \textsl{tile} όπως επίσης και την ασυνέχεια του αντίστοιχου χώρου στην κύρια μνήμη ώστε να χρησιμοποιήσουμε τον μηχανισμό των \textsl{λιστών \ac{DMA}} (\textsl{\ac{DMA} lists}) για την μεταφορά των δεδομένων προς την \ac{LS}. Έτσι, πολλαπλές συνεχόμενες αιτήσεις \ac{DMA} αντικαθίστανται από μία αίτηση για την μεταφορά μίας λίστας \ac{DMA}. Όσον αφορά στην μεταφορά δεδομένων από την \ac{LS} προς την κύρια μνήμη, η ασυνέχεια του χώρου της \ac{LS} από τον οποίο μεταφέρονται δεδομένα δεν επιτρέπει την χρήση των \textsl{λιστών \ac{DMA}}.

\section{Εκμετάλλευση της Ομοιότητας μεταξύ Διαδοχικών Frames}
\label{section:sect55}
\indent
Το κυρίως κίνητρο για την βελτιστοποίηση του αλγορίθμου είναι η εφαρμογή του σε συστήματα \textsl{videoconferencing}. Όπως γνωρίζουμε, ένα \textsl{video} είναι μία ακολουθία από \textsl{frames}. Βάσει αυτού, μία προφανής βελτιστοποίηση σε αλγοριθμικό επίπεδο είναι η εκμετάλλευση της ομοιότητας που εμφανίζεται μεταξύ διαδοχικών \textsl{frames}. Σε αυτή την περίπτωση, δεν απαιτείται η εκ νέου εφαρμογή του αλγορίθμου σε \textsl{tiles} διαδοχικών \textsl{frames} τα οποία δεν μεταβάλλονται σε μεγάλο βαθμό.\newline \indent
Για την εύρεση των διαφορών μεταξύ διαδοχικών \textsl{tiles} είναι απαραίτητη η εφαρμογή ενός είδους \textsl{εκτίμηση κίνησης} (\textsl{Motion Estimation}). Οι αλγόριθμοι \textsl{εκτίμησης κίνησης} χρησιμοποιούνται σε πρότυπα για συμπίεση/αποσυμπίεση βίντεο (π.χ. \textsl{H.264, AVS}) και τείνουν να έχουν μεγάλη υπολογιστική πολυπλοκότητα και απαιτήσεις σε εύρος ζώνης. Επί παραδείγματι, η πλειονότητα των αλγορίθμων απαιτεί την μεταφορά του προηγουμένου \textsl{frame} πολλαπλές φορές λόγω της επικάλυψης που υπάρχει μεταξύ των διαφόρων \textsl{παραθύρων ενδιαφέροντος}. Δεδομένης της μικρής χωρητικότητας της \ac{LS} των \acp{SPE}, η πιθανότητα για σημαντική επιβράδυνση της εφαρμογής λόγω των πολλαπλών μεταφορών από την κύρια μνήμη στην \ac{LS} είναι σημαντική. Επίσης, λόγω της μεγάλης υπολογιστικής πολυπλοκότητας είναι πολύ πιθανό ο χρόνος εκτέλεσης του αλγορίθμου \textsl{εκτίμησης κίνησης} να ξεπερνά κατά πολύ τον χρόνο που εξοικονομείται από την μη εφαρμογή του αλγορίθμου διόρθωσης στα αντίστοιχα \textsl{tiles}.

