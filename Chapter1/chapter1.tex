\chapter{Εισαγωγή}
\label{chapter:chap1}

\section{Περιγραφή του Προβλήματος}
\label{section:sect11}
\indent
Οι κωδικοποιητές βίντεο είναι αναγκαίοι για να μειώσουν τον τεράστιο αριθμό δεδομένων που έχει ένα βίντεο. Σε οπτικά σήματα με ισχύ μεγαλύτερη των \si{35}{dB} δεν είναι εύκολο να παρατηρήθει διαφορά από το ανθρώπινο μάτι. Συνεπώς, επιλέγεται να εισαχθεί ομοιόμορφος θόρυβος για να επιτευχθεί μεγαλύτερος λόγος συμπίεσης θυσιάζοντας την ποιότητα.
Η κωδικοποίηση βίντεο υλοποιείται τόσο σε επίπεδο λογισμικού αλλά και υλικού. Ενώ οι προσωπικοί υπολογιστές συνήθιζαν να χρησιμοποιούν λογισμικό για την αναπαραγωγή βίντεο, τα τελευταία χρόνια η αύξηση των αναλύσεων του βίντεο (1080p,4K) καθιστά αδύνατη ή πολύ δαπανηρή την αποκωδικοποίηση ενός συμπιεσμένου βίντεο από έναν επεξεργαστή γενικής χρήσης. Έτσι συναντάμε υλικό ειδικά σχεδιασμένο (hardware accelerators) για κωδικοποίηση/αποκωδικοποίηση βίντεο που βοηθά τον κύριο επεξεργαστή.

\begin{table}[h!]
    \begin{center}
        \begin{tabular}{| l | l | l | l |}
        \hline
        Έτος  & Κωδικοποιητής   & Διάφορες Χρήσεις      &      Bitrate(Mbps) (720x480)  \\ \hline
        1993    & MPEG-1        &       VCD             &       7                		\\ \hline
        1995    & MPEG-2        &       DVD             &       6                		\\ \hline
        1999    & MPEG-4        &    DivX,XVid          &       5                		\\ \hline
        2003    & H.264         & BluRay,DVB-TS         &       4                		\\ \hline
        2013    & H.265         & next generation H.264 &       2                		\\ \hline
        \hline
        \end{tabular}
    \end{center}

    \caption{Κυριότεροι κωδικοποιητές βίντεο. \cite{wiki:codecs}}
    \label{table:listofcodecs}
\end{table}

\indent
Σήμερα όπου υπάρχει βίντεο υπάρχει και κωδικοποίηση. Όπως φαίνεται στον πίνακα~\ref{table:listofcodecs} με το πέρασμα του χρόνου οι κωδικοποιητές γίνονταν ολοένα και πιο αποτελεσματικοί αυξάνοντας όμως κατά πολύ την πολυπλοκότητα τους. Πλέον, υπάρχει η δυνατότητα με λίγα δεδομένα να έχουμε βίντεο καλής ποιότητας σε πολύ μεγάλες αναλύσεις. Ενδεικτικά για ένα ασυμπίεστο βίντεο ανάλυσης DVD (720x480) με διάρκεια 20sec και ρυθμό ανανέωσης \si{30}{Hz} απαιτείται χώρος 296ΜΒ. Κάνοντας συμπίεση με τον H.264 προκύπτει ένα αρχείο 0.8ΜΒ.

\section{Συμβολή της Εργασίας}
\label{section:sect12}
\indent
Στην εργασία αυτή γίνεται χρήση της κβαντοποίησης διανυσμάτων στον κωδικοποιητή H.264. Είναι μια τεχνική συμπίεσης δεδομένων με απώλειες (lossy compression), η οποία έχει δοκιμαστεί ελάχιστα παλαιότερα στο βίντεο και φαίνεται πως μπορεί να δώσει λύσεις στον λόγο συμπίεσης καθώς επίσης και στην μείωση της πολυπλοκότητας του αποκωδικοποιητή. Ο τρόπος που μειώνεται η πολυπλοκότητα μπορεί να μετατρέψει ακριβά ενεργοβόρα κυκλώματα σε φθηνές μνήμες χωρίς απαιτήσεις ισχύος.

\section{Διάρθρωση της Διπλωματικής Εργασίας}
\label{section:sect13}


\indent
Στο Κεφάλαιο~\ref{chapter:chap2} παρουσιάζεται το ψηφιακό βίντεο, τις μορφές αποθηκεύσης του, πώς οι κωδικοποιητές το αντιμετωπίζουν και ποιές είναι οι κύριες τεχνικές που χρησιμοποιούνται για την συμπίεση του.\newline

\indent
Στο Κεφάλαιο~\ref{chapter:chap3} γίνεται μία σύντομη εισαγωγή στον τομέα της Θεωρίας Πληροφοριών. Είναι αναγκαίο να εξηγηθεί τι είναι η πληροφορία καθώς και η εντροπία αυτής. Επίσης θα εξηγηθεί αναλυτικά ο κύριος αλγόριθμος clustering k-means καθώς και οι όποιες βελτιστοποίησες προέκυψαν στην πορεία της παρούσας διπλωματικής.\newline

\indent
Στο Κεφάλαιο~\ref{chapter:chap4} παρουσιάζονται τα βήματα που έγιναν για την παραγωγή των codebooks και την επιπρόσθετη κατηγοριοποίηση τους με στόχο την μείωση της εντροπίας.\newline

\indent
Στο Κεφάλαιο~\ref{chapter:chap5} παρουσιάζεται η επέμβαση που έγινε στον H.264 έτσι ώστε να χρησιμοποιεί τα codebooks για να κάνει την κωδικοποίηση/αποκωδικοποίηση.\newline 

\indent
Στο Κεφάλαιο~\ref{chapter:chap6} παρουσιάζονται τα αποτελέσματα του VQ-H.264.\newline

\indent
Στο Κεφάλαιο~\ref{chapter:chap7} συγκρίνονται τα αποτελέσματα του VQ-H.264 με τον JM-H.264.\newline